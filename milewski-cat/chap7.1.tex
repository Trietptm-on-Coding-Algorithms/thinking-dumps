\setcounter{section}{7}
\setcounter{subsection}{0}
\subsection{Bifunctors}

We start by studying if ADTs form are functors. Product type in haskell is
actually a type constructor of two argument. \ie. \code{(a,b)} can be written as
\code{(,) a b}.

It's not hard to see that \code{(,) a}, \ie. fix on one parameter \code{a}, is a
functor from \code{b} to \code{(a, b)}.

\begin{center}
\begin{tikzcd}[math mode=false, sep=huge]
  \code{(e,x)} \rar{\code{fmap f}} & \code{(e,y)} \\
  \code{x} \uar{\code{(e,)}} \rar{\code{f}} & \code{y} \uar{\code{(e,)}}
\end{tikzcd}
\end{center}

Now think, why don't we make a functor type that has two arguments, so we can
make \code{(,)} a functor on both arguments. In math, we could try to define
a category that formalize product of two categories instead of inventing a
something new, like a new kind of functor, for that purpose.
But what will this category look like? Well it's basically just product of the
hom sets. For categories $C$ and $D$, we define $C \times D$ as follows:

\begin{itemize}
\item For each pair of objects $c\in C$ and $d\in D$, there is an object $(c,d)
  \in C\times D$;
\item For each pair of arrows $f=c\mapsto c'$ in $C$ and $g
  = d\mapsto d'$ in $D$, there is an arrow $(f, g)=(c,d)\mapsto (c',d')$ in $C\times D$;
\item Composition: $(f', g') \circ (f, g) = (f'\circ f, g'\circ g)$;
\item Identity: $\id_{(a,b)}=(\id_a, \id_b)$.
\end{itemize}

So the functor that takes two arguments is just a functor $C\times
D \to E$. This kind of functors is called a bifunctor.

In Haskell, bifunctor is defined as:

\begin{lstlisting}
class Bifunctor b where
  bimap :: (a -> a') -> (b -> b') -> f a b -> f a' b'
\end{lstlisting}

And this is how \code{(,)} (product) and \code{Either} (sum) implemented as bifunctors:

\begin{lstlisting}
instance Bifunctor (,) where
  bimap f g (a,b) = (f a, g b)

instance Bifunctor Either where
  bimap f g (Left a)  = Left (f a)
  bimap f g (Right a) = Right (g a)
\end{lstlisting}

In a cartesian category $C$, in which there are products for every pair of
objects, then this ``product'' is a bifunctor $C\times C \to C$. This also
applies to ``co-product'' categories. Here's how it works:

\begin{center}
\begin{tikzcd}[sep=huge]
  & a \times b \ar{ld}[below]{\pi_1} \ar{rd}[below]{\pi_2} & \\
  a \ar{dd}{f} & & b \ar{dd}{g} \\
  & a' \times b' \ar{ld}{\pi_1'} \ar{rd}[below]{\pi_2'} & \\
  a' & & b'
\end{tikzcd}
\end{center}

So we have this diagram. For each $a, b \in C$, by definition, we have its
product $a\times b$ which projects ($\pi_1, \pi_2$) to $a$ and $b$. And for the
morphisms, we have \cd{a \rar{f} \& a'} and \cd{b \rar{g} \& b'}. In order for
the product ($\times : C\times C \to C$), to be a bifunctor, we need to show it
is possible to lift the pair $f, g$ into something that
\cd[normal]{a\times b \rar{f\times g} \& a' \times b'}. \ie there exists a
morphism from $a\times b$ to $a'\times b'$ on the diagram above.

\begin{center}
\begin{tikzcd}[sep=huge]
  & a \times b
  \ar{ld}[below]{\pi_1} \ar[red]{lddd}[left]{f \circ \pi_1}
  \ar{rd}[below]{\pi_2} \ar[red]{rddd}[right]{g \circ \pi_2}
  \ar[dashed]{dd}[below]{f\times g}
  & \\
  a \ar{dd}{f} & & b \ar{dd}{g} \\
  & a' \times b' \ar{ld}{\pi_1'} \ar{rd}[below]{\pi_2'} & \\
  a' & & b'
\end{tikzcd}
\end{center}

Here's how we show that. Since we have $\pi_1$ and $f$, they will compose into
$f \circ \pi_1$, same for $\pi_2$ and $g$. By definition of the univeral
construction of products, we can see
that $a'\times b'$ is the product on $a'$ and $b'$. Therefore, there must exist a unique
factorialization from any pair $a\times b$ to $a'\times b'$ (shown as the dashed arrow).
This morphism is what we are looking for and therefore we have shown the
existence of $\times$ bifunctor.

In haskell, this is not an issue. As the implementation above, we just apply $f$
to $a$ and $g$ to $b$, then we will form $(f~a, g~b)$. But this is not enough
general for Math, we need to think out of just the $\Hask$ category. So the
above diagram shows how it work generally in any category.