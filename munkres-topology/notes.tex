\documentclass{report}

\usepackage[utf8]{inputenc}

\usepackage{imakeidx}
\makeindex[columns=2,title=Alphabetical Index]

\usepackage{tikz}
\usetikzlibrary{fit,shapes}
\tikzstyle{point}=[draw,fill=black,circle,inner sep=0pt,minimum size=2pt]

\usepackage{amsmath}
\usepackage{amssymb}
\let\oldemptyset\emptyset
\let\emptyset\varnothing

\usepackage{hyperref}
\newcommand*{\link}[1]{\href{#1}{(\underline{link})}%
  \footnote{\url{#1}}}

\usepackage{xspace}
\usepackage{float}
\usepackage{enumitem}

\usepackage{amsthm}
\newtheorem*{remark}{Remark}

\theoremstyle{definition}
\newtheorem{theorem}{Theorem}[chapter]
\newtheorem{proposition}[theorem]{Proposition}
\newtheorem{definition}[theorem]{Definition}
\newtheorem{example}{Example}[theorem]
\newtheorem{property}[theorem]{Property}
\newtheorem{corollary}[theorem]{Corollary}

\newcommand{\defn}[1]{\textbf{#1}\label{#1}\index{#1}}
\newcommand{\set}[1]{\ensuremath\left\{#1\right\}}

\newcommand{\ie}{\textit{i.e.}\xspace}
\newcommand{\aka}{\textit{a.k.a.}\xspace}
\newcommand{\eg}{\textit{e.g.}\xspace}

\newcommand{\RA}{\ensuremath\Rightarrow}
\newcommand{\LA}{\ensuremath\Leftarrow}

\newcommand{\ZZ}{\mathbb{Z}}
\newcommand{\RR}{\mathbb{R}}
\newcommand{\CC}{\mathbb{C}}
\newcommand{\QQ}{\mathbb{Q}}

\newcommand{\intersect}{\cap}
\newcommand{\union}{\cup}
\newcommand{\Intersect}{\bigcap}
\newcommand{\Union}{\bigcup}

\newcommand{\Int}{\operatorname{Int}}
\newcommand{\interior}[1]{#1^\circ}
\newcommand{\closure}[1]{\overline{#1}}

\newcommand{\ex}{\ensuremath\exists}
\newcommand{\ep}{\epsilon}
\newcommand{\inv}{\ensuremath^{-1}}

\begin{document}


\title{Notes on James R. Munkres' Topology (2E)}
\author{Shou}
\date{\today}

\maketitle
\tableofcontents

\def\T{\mathcal{T}}

\setcounter{chapter}{-1}
\begin{chapter}{Structure and reading plans}
  Ch 1-8 is the part I, mainly for common topology. The part II includes
  ch 9-14, that depends on ch 1-4, is about algebraic topology.

  My plan is to read through ch 1-4 very quickly, within a weekend, and
  then I will start reading ch 9+ simultaneously with W.S.Massey's
  Algebraic topology: An induction.

  Finally I wish I could finish all ch 1-8 and also some parts after ch 9.
\end{chapter}

\begin{chapter}{Set theory and logic}
  \begin{definition}{\defn{Order relation}}
    rel $C$ on set $A$ is called \emph{order relation} if
    \begin{enumerate}
    \item comparability, \aka totality for all non-eq elements, \ie
      $\forall x,y\in A, x\not=y\RA x C y\lor y C x$
    \item non-refl, \ie $\forall x, \neg(x C x)$
    \item trans, \ie $\forall x C y\land y C z, x C z$
    \end{enumerate}
    (\aka \defn{Linear order}, \defn{Simple order})
  \end{definition}
  \begin{remark}
    This relation is not the same as Linear order on Wikipedia
    \link{https://en.wikipedia.org/wiki/Total_order}. This order is
    actually the strict version of the Total order on wikipedia, \ie
    has non-refl property.
  \end{remark}

  \begin{definition}{\defn{Open interval}}
    if $X$ is a set and $<$ is an order rel, and if $a<b$ we use
    notation $(a,b)$ to denote $\set{x\in X\mid a<x<b}$, called
    \emph{open interval}.

    If $(a,b) = \emptyset$, then $a$ is called \emph{immediate
      precessor} of $b$ and $b$ called \emph{immediate successor} of
    $a$.
  \end{definition}
  \begin{remark}
    It makes more sense on $X$ is a discrete set. Since if
    $(a,b)$ is an open interval in $\RR$, $(a,b) = \emptyset
    \RA a = b$ which makes no sense on $a$ as an immediate precessor
    of $b$.
  \end{remark}

  \begin{definition}{\defn{Order type}}
    if $A$ and $B$ are two sets with $<_A$ and $<_B$. We say that $A$
    and $B$ have same \emph{order type} if $\ex f : A \to B$ that
    preserves order, \ie $$a_1 <_A b_1 \RA f(a_1) <_B f(b_1)$$
  \end{definition}
  \begin{remark}
    It's just a generalization of monotone function.
  \end{remark}

  \begin{definition}{\defn{Dictionary order relation}}
    if $A$,$B$ are two sets with $(<_A,<_B)$, defn an order for
    $A\times B$ by defining $$a_1\times b_1 < a_2\times b_2$$
    if $a_1<_A a_2$, or if $a_1=a_2\;\land\;b_1<_B b_2$.
  \end{definition}

  \begin{definition}{\defn{LUB property}/\defn{GLB property}}
    For $A$ and $<_A$, we say $A$ has \emph{LUB property} if
    $$\forall A_0\subset A,A\not=\emptyset\land\ex \text{upper bound for }
    A_0 \RA \ex \text{lub$\{A_0\}$}\in A$$
  \end{definition}
  \begin{example}
    $A=(-1,1)$. \eg $X=\set{1-\frac{1}{n}\mid n\in\ZZ^+}$ does not have an
    upper bound, thus vacuously true. $\set{-\frac{1}{n}\mid
      n\in\ZZ^+}$ has upper bound of any number in $[0,1)\subset A$, and
    $\operatorname{lub}(X)=0\in (-1,1)$.
  \end{example}
  \begin{example}
    Counterexample. $A=(-1,0)\cup(0,1)$. $\set{-\frac{1}{n}\mid n\in\ZZ^+}$ has
    upper bound of any $(0,1)\subset A$, while
    $\operatorname{lub}(X)=0\not\in A$.
  \end{example}
  \begin{remark}
    The completeness property of $\RR$ as an axiom derives this property.
  \end{remark}

  \begin{property}{$\RR$ field}

    \textbf{Algebraic properties}
    \begin{enumerate}
    \item assoc: $(x+y)+z=x+(y+z); (xy)z=x(yz)$
    \item comm: $x+y=y+x; xy=yx$
    \item id: $\ex!0, x+0=x$; $\ex!1,x\not=0\RA x1=x$
    \item inv: $\forall x,\ex!y,x+y=0$; $\forall x\not=0,\ex!y,xy=1$
    \item distr: $x(y+z)=xy+xz$
    \end{enumerate}

    \textbf{Mixed algebraic and order property}
    \begin{enumerate}[resume]
    \item $x>y\RA x+z>y+z$; $x>y\land z>0\RA xz>yz$
    \end{enumerate}

    \textbf{Order properties}
    \begin{enumerate}[resume]
    \item $<$ has LUB property
    \item $\forall x<y,\ex z, x<z\land z<y$
    \end{enumerate}
  \end{property}

  1-6 make $\RR$ a field. 1-6 + 7 make $\RR$ an ordered field. 7-8
  makes $\RR$, called by topologists, a \defn{Linear continuum}.

  \begin{theorem}{Well ordering property}
    $\ZZ^+$ has \emph{Well-ordering property}. \ie Every nonempty
    subset of $\ZZ^+$ has a smallest element.
  \end{theorem}
  \begin{proof}
    We first prove that for each $n\in\ZZ^+$, the following statement
    holds: Every nonempty subset of $\set{1,\ldots,n}$ has a smallest
    element.

    Let $A$ be the set of all postive integers $n$ for which this
    theorem holds. Then $A$ contains 1, since if $n=1$, the only
    possible subset is $\set{1}$ itself. Then suppose $A$ contains $n$,
    we show that it contains $n+1$. So let $C$ be a nonempty subset of
    the set $\set{1,\ldots,n+1}$. If $C$ consists of the single
    element $n+1$, then that element is the smallest element of
    $C$. Otherwise, consider the set $C\cap\set{1,\ldots,n}$, which is
    nonempty. Because $n\in A$, this set has a smallest element, which
    will automaticallly be the smallest element of $C$ also. Thus $A$
    is inductive, so we conclude that $A=\ZZ^+$; hence the statement
    is true for all $n\in\ZZ^+$.

    Now we prove the theorem. Suppose that $D$ is a nonempty subset of
    $\ZZ^+$. Choose an element $n$ of $D$. Then the set $A=D\cap[n]$
    is nonempty, so that $A$ has a smallest element $k$. The element
    $k$ is automatically the smallest element of $D$ as well.
  \end{proof}
  \begin{remark}
    I don't really understand the second part of this proof. By
    {\textup{https://proofwiki.org}}, Principle of Mathematical
    Induction, Well-Ordering Principle, and Principle of Complete
    Induction are logically equivalent.
    \link{https://proofwiki.org/wiki/Equivalence_of_Well-Ordering_Principle_and_Induction\#Final_assembly}
  \end{remark}

  \begin{definition}{\defn{Cartesian product}}
    Let $\set{A_1,\ldots,A_m}$ be a faimily of sets indexed with the
    set $\set{1.\ldots,m}$. Let $X=A_1\cup\dots\cup A_m$. We define
    \emph{cartesian product} of this indexed family, denoted by
    $$\prod_{i=1}^mA_i\;\;\text{or}\;\;A_1\times \dots \times A_m,$$
    to be the set of all $m$-tuples $(x_1,\ldots,x_m)$ of elements of
    $X$ such that $x_i\in A_i$ for each $i$.
  \end{definition}
  \begin{remark}
    Indexing function $f:J\to\mathcal{A}$ is surjective but not
    necessarily injective.
  \end{remark}

  \begin{definition}{\textbf{$\omega$-tuple}}
    An \emph{$\omega$-tuple} of elements of set $X$ to be a function
    $$x:\ZZ^+\to X,$$
    \aka \emph{sequence}, or a \emph{infinite sequence}.
  \end{definition}

  \begin{theorem}
    $\set{0,1}^\omega$ is uncountable. (let $X=\set{0,1}$ in the proof.)
  \end{theorem}
  \begin{proof}
    We show that given any function $g:\ZZ^+\to X^\omega$, $g$ is not
    surjective. Four this purpose, let us denote $g(n)$ as
    $(x_{n1},x_{n2},\ldots,x_{nn},\ldots,x_{n\omega})$, where each $x_{ij}$ is
    eather 0 or 1. Then we define any element $y=(y1,\ldots,y_\omega)$
    of $X^\omega$ by letting
    \begin{equation*}
      y_n=
      \begin{cases}
        0 & \text{if $x_{nn}$ = 1}, \\
        1 & \text{if $x_{nn}$ = 0}.
      \end{cases}
    \end{equation*}
    $y$ will differ $g(n)$ for all $n$ by a digit. Therefore
    $y\not\in \operatorname{Im}(g)$.
  \end{proof}
  \begin{remark}
    Note this proof is similar to the proof of uncountableness of
    $[0,1)$ using the vast digit array.
  \end{remark}
  \begin{remark}
    $\set{0,1}^\omega
    \simeq [0,1)$ by $f(a_1,a_2,\ldots)=\sum_{i=1}^\infty{a_i
      2^{-i}}$. (\ie binary decimals). Then we can use the conclusion
    of the uncountableness of $[0,1)$ to prove this directly.
  \end{remark}
  \begin{remark}
    Think of picking a subset of $\ZZ^+$, for each $i\in\ZZ^+$ present in
    the subset, set $a_i=1$, otherwise $a_i=0$. Thus
    $\set{0,1}^\omega$ is just isomorphic to the power set
    $2^{\ZZ^+}$. By cantor's theorem, there is not surjection
    $f : \ZZ^+\to 2^{\ZZ^+}$.
  \end{remark}

  \begin{theorem}
    There is not surjective map $g:A\to 2^A$ for all set $A$. Proof:~\link{https://proofwiki.org/wiki/Cantor's_Theorem}
  \end{theorem}


  \begin{theorem}{\defn{Principle of recursive definition}}
    Let $A$ be a set; let $a_0$ be an element of $A$. Suppose $\rho$
    is a function that assigns, to each function $f$ mapping a
    nonempty section of the positive integers into $A$, an element of
    $A$. Then there exists a unique function
    $$h : \ZZ^+ \to A$$
    such that
    \begin{align*}
      h(1) &= a_0, \\
      h(i) &= \rho(h|\set{1,\ldots,i-1})\;\;\text{for $i>1$}
    \end{align*}
    The formula is called a \emph{recursion formula} for $h$.
  \end{theorem}
  \begin{remark}
    I'm not very clear about this definition. I think the point of
    this definition is to indicate that there is a UNIQUE function
    satisified a recursive definition.
  \end{remark}

  \begin{theorem}
    The following statements about set $A$ are equivalent:
    \begin{enumerate}
    \item There exists an \emph{injective}, not necessarily surjective
      (of course), function $f : \ZZ^+ \to A$.
    \item There exists a bijection of $A$ to a propert subset of $A$.
    \item $A$ is infinite.
    \end{enumerate}
  \end{theorem}
  \begin{example}\mbox{}
    \begin{enumerate}
    \item $f : \ZZ^+\hookrightarrow\RR$.
    \item $f : \RR \to (-1,1)$ by $f(0)=0, f(a)=\frac{1}{a}$.
    \item $\RR$ is infinite.
    \end{enumerate}
  \end{example}

  \begin{definition}{\defn{Axiom of choice}}
    Given a collection $\mathcal{A}$ of disjoint nonempty sets (\ie
    $\forall x,y\in\mathcal{A}, x\cap y=\emptyset$), there
    exists a set $C$ consisting of exactly one element from each
    element of $\mathcal{A}$; that is, $C\subset\bigcup\mathcal{A}$ and for
    each $A\in\mathcal{A}$, the set $C\cap A$ contains a single
    element.

    The set $C$ can be thought of as having been obtained by choosing
    one element from each of the sets in $\mathcal{A}$.
  \end{definition}

  \begin{theorem}\defn{Well-ordering theorem}
    Every set is well-orderable. (assuming axiom of choice)
  \end{theorem}
  \begin{remark}
    This theorem is rather against intuition. As I checked briefly on
    wikipedia and proofwiki, the proof
    \link{https://proofwiki.org/wiki/Well-Ordering_Theorem} is done
    using the axiom of choice and the Princple of Transfinite
    Induction. I can't comprehend the proof since I have no related
    knowledge about transfinite induction schema. The result is also
    known as \defn{Zermelo's
      Theorem}\link{https://proofwiki.org/wiki/Zermelo\%27s_Theorem_(Set_Theory)},
    which is usually worded: Every set of cardinals is well-ordered
    with respect to $\leq$. This is weird to me since doesn't it imply
    that cardinals are equivalent to ordinals?
  \end{remark}

  \begin{definition}\defn{Strict partial order}
    A set $A$, a relation $\prec\;\subset A \times A$ is a \emph{strict
      partial order} if for all elements $a,b,c\in A$,
    \begin{enumerate}
    \item Non-refl: $\neg(x \prec x)$
    \item Trans: $x\prec y\land y\prec z \RA x\prec z$
    \end{enumerate}
  \end{definition}
  \begin{remark}
    Compare to (non-strict) partial order, strict partial order does
    not have reflexivity property ($x\leq x$).
  \end{remark}
  \begin{remark}
    Compare to a simply order relation, this order relation does not
    require totality (or comparability).
  \end{remark}

  \begin{theorem}\defn{The maxium principle}
    A set $A$, a strict partial order $\prec$ on $A$. There exists a
    maxiaml simply ordered subset B.

    Said differently, there exists a subset $B$ of $A$ such that $B$
    is simply ordered by $\prec$ and such that no subset of $A$ that
    properly contains $B$ is simply ordered by $\prec$.
  \end{theorem}
  \begin{remark}
    Think in this way. Draw a acyclic directed graph for $A$: for
    for an element in $A$ draw a vertex, for each pair of vertices,
    draw an edge from $a$ to $b$ iff $a\prec b$ is minimal, \ie
    $\neg\ex c: a\prec c\prec b$. Then we can say
    $a\prec b$ if $a$ is connected to $b$ by the transitivity
    property. This thereom actually says that there exists a maximal
    path in the graph.
  \end{remark}

  \begin{definition}\defn{Upper bound} and \defn{Maximal element}
    Let $A$ be set and $\prec$ be strict partial order on
    $A$. $B\subset A$. An \emph{upper bound} on $B$ is $c\in A$ such
    that $\forall b\in B:b=c\lor b\prec c$. A \emph{maximal element}
    of $A$ is an element $m\in A$ such that $\neg\ex a,m\prec a$.
  \end{definition}
  \begin{remark}
    When we talk about upper bounds of $A$, we are under implication
    of a subset $A$ of a ordered set.
  \end{remark}

  \begin{theorem}\defn{Zorn's Lemma}
    Let $A$ be strictly partially ordered set. If every simply ordered
    subset of $A$ has an upper bound in $A$, then $A$ has a maximal
    element.
  \end{theorem}
  \begin{remark}
    This is a consequence of the maximum principle. It's easy to
    verify the maximum element is the upper bound of the maximal
    simply ordered subset.
  \end{remark}
\end{chapter}

\begin{chapter}{Topological spaces and continuous functions}
  \begin{definition}\defn{Topology}
    A \emph{topology} on a set $X$ is a collection $\mathcal{T}$ of
    subsets of $X$ (\ie $\mathcal{T}\subset 2^X$) having the following properties:
    \begin{enumerate}
    \item $\emptyset \in \mathcal{T}$ and $X\in\mathcal{T}$,
    \item The union of the elements of any subcollection of $\mathcal{T}$ is in
      $\mathcal{T}$,
    \item The intersection of elements of any \underline{finite}
      subcollection of $\mathcal{T}$ is in $\mathcal{T}$.
    \end{enumerate}
  \end{definition}
  \begin{remark}
    After investigation, I consider there is no special meaning for a
    collection differing from a
    set. (ref. \link{http://math.stackexchange.com/a/173002},
    \link{https://proofwiki.org/wiki/Definition:Topology/Definition_1})
  \end{remark}

  \begin{definition}\defn{Topological space}
    A \emph{topological space} is an ordered pair $(X,\mathcal{T})$
    consisting of a set $X$ and a topology $\mathcal{T}$ of
    $X$. Often we omit specfic mention of $\mathcal{T}$ if no
    confusion will arise.
  \end{definition}

  \begin{definition}\defn{Open set}
    If $X$ is a topological space, we say that $U\subset X$ is
    an \emph{open set} of $X$ if $U \in \mathcal{T}$. Using this
    terminology, one can say a topological space is a set $X$ together
    with a set of open sets. Thus the defintion of an open set is the
    same as topology: $\emptyset$ and $X$ both open, arbitrary
    unions and finite intersections of open sets are open.
  \end{definition}

  \begin{definition}
    If $\mathcal{T}=2^A$, then $X$ is called a \defn{Discrete
      topology}; \\
    If $\mathcal{T}=\set{\emptyset, X}$, then $X$ is called a
    \defn{Trivial topology}, or \defn{Indiscrete topology}.
  \end{definition}
  \begin{definition}\defn{Finite complement topology}
    Let $X$ be a set, $\mathcal{T}_f$ be the collection of all subsets
    $U$ of $X$ such that $X-U$ either is finite or is all of $X$. Then
    $\mathcal{T}_f$ is a topology on $X$, called the \emph{finite
      complement topology}. Both $X$ and $\emptyset$ are in
    $\mathcal{T}_f$ since $X-X$ is finite and $X-\emptyset$ is all of
    $X$.

    Now we show $\mathcal{T}_f$ is a topology. If $\set{U_a}$ is an
    indexed family of nonempty elements of $\T_f$, to show $\bigcup U_a$
    is in $\T_f$, we compute
    $$X-\bigcup U_a = \bigcap(X-U_a)$$
    Which is finite since $X-U_a$ is finite for all $U_i\in\T_f$. To show
    $\bigcap U_i\in\T_f$, we compute
    $$x-\bigcap_{i=1}^n U_i=\bigcup_{i=1}^n(X-U_i)$$
    Note by definition of topology we only need to check finite
    intersection. Then the finite number of unions of finite sets is
    finite thus finite $\bigcup U_i\in\T_f$.
  \end{definition}

  \begin{definition}\defn{Finer}, \defn{Strictly
      finer}, \defn{Coarser}, \defn{Strictly coarser}, and
    \defn{Comparable}
    \begin{table}[H]
      \centering
      \begin{tabular}{l|l|l}
        & set term & topology term \\\hline
        $\T'\supset \T$ & superset & finer, larger, stronger \\
        $\T'\supsetneq \T$ & proper superset(?) & strictly finer  \\
        $\T'\subset \T$ & subset & coarser, larger, weaker   \\
        $\T'\subsetneq \T$ & proper subset & strictly coarser \\
        $\T'\subset \T$ or $\T\subset T'$ & - & comparable
      \end{tabular}
    \end{table}
  \end{definition}
  \begin{remark}
    This is a bit not straightforward at the first sight. Remember
    that the essential of topology is its structure. The more open sets
    we have in a topology, the more fine it is.
  \end{remark}

  \def\B{\mathcal{B}}
  \begin{definition}\defn{Basis}
    A \emph{basis} for a topology $\B\subset 2^X$ (called basis
    elements) such that
    \begin{enumerate}
    \item $\forall x\in X : \exists B\in\B : x\in B$
    \item $\forall x\in B_1 \cap B_2 \RA \ex B_3 \in \B: x\in B_3 \land
      B_3\subset B_1\cap B_2$
    \end{enumerate}

    If $\B$ is a basis, we define \emph{topology $\T$ generated by
      $\B$} to be: we say $U\subset X$ is open set (\ie $U\in\T$), if
    $\forall x\in U: \exists B\in\B: x\in B \land B\subset U$.
  \end{definition}
  \begin{remark}
    The concept of a basis of a topology is rather abstract. I think
    of it in this way. Since we are dealing with intersection in the
    definition of basis, we want to think from top to bottom, in other
    words, from larger sets, to their intersections.
  \end{remark}
  \begin{example}
    Think of this example. In the beginning we have $X=\set{a,b,c}$
    and ${\B=\set{B_1,B_2}}$ where $B_1=\set{a,b}$ and
    $B_2=\set{b,c}$.

    \begin{figure}[H]
      \centering
      \tikz{
        \node (a) [point,label={[name=la] a}] {};
        \node (b) [point,right of=a,label={[name=lb] b}] {};
        \node (c) [point,right of=b,label={[name=lc] c}] {};
        \node (B1) [ellipse,fit=(a)(la)(b),draw,label=below:$B_1$] {};
        \node (B2) [ellipse,fit=(b)(lc)(c),draw,label=below:$B_2$] {};
      }
    \end{figure}

    In this way $\forall x\in X :\ex B\in \B: x\in B$. Now we try to
    satisfy the second condition and we will find that
    $b\in B_1\cap B_2$ is not in any basis element who is a subset of
    $B_1\cap B_2$. Now we add it as follows:

    \begin{figure}[H]
      \centering
      \tikz{
        \node (a) [point,label={[name=la] a}] {};
        \node (b) [point,right of=a,label={[name=lb] b}] {};
        \node (c) [point,right of=b,label={[name=lc] c}] {};
        \node (B1) [ellipse,fit=(a)(la)(b),draw,label=below:$B_1$] {};
        \node (B2) [ellipse,fit=(b)(lc)(c),draw,label=below:$B_2$] {};
        \node (B3) [ellipse,fit=(b)(lb),draw=blue,label={[yshift=-7pt,blue] below:$B_3$}] {};
      }
    \end{figure}

    Now with $\B=\set{B_1,B_2,B_3}$ all above two criterion are
    satisified. Therefore $\B$ is a basis for the topology $\T$:

    \begin{figure}[H]
      \centering
      \tikz{
        \node (a) [point] {};
        \node (b) [point,right of=a] {};
        \node (c) [point,right of=b] {};
        \node (b0) [fit=(b),circle,draw] {};
        \node (ab) [fit=(a)(b0),ellipse,draw,inner sep=0] {};
        \node (bc) [fit=(c)(b0),ellipse,draw,inner sep=0] {};
        \node (abc) [fit=(ab)(bc),ellipse,draw,inner sep=0] {};
      }
    \end{figure}
  \end{example}

  \begin{example}
    Continuing above example. We now verify that $\T$ is generated by
    $\B$ by checking all open sets $U\in\T$ with the rule specified.
    Actually \\
    ${\T=\set{\emptyset, \set{a,b}, \set{b,c}, \set{b},
        \set{a,b,c}}}$. I will only show how $\set{a,b}\in\T$ and how
    $\set{b}\in\T$ and how $\set{a,c}\not\in\T$.

    \begin{enumerate}
    \item $U=\set{a,b}\in\T$: For $a\in U$, take $B=\set{a,b}$, then
      $B\subset U$; the same works for $b$. Therefore $U$ is an open
      set.
    \item $U=\set{b}\in\T$: For $b\in U$, take $B=\set{b}$, then
      $B\subset U$. Therefore $\set{b}$ is an open set. Note that we
      cannot take $B=\set{a,b}$, since $\set{a,b}\not\subset \set{b}$.
    \item $U=\set{a,c}\not\in\T$: For $a\in U$, we cannot find a
      $B\in\B$ that contains $a$ and is a subset of $\set{a,c}$. Since
      the only possible subsets of $\set{a,c}$ are $\set{\emptyset,
        \set{a}, \set{c}, \set{a,c}}$ and neither is a basis
      element. Therefore $\set{a,c}$ is not an open set.
    \end{enumerate}
  \end{example}
  \begin{remark}
    Let $\B$ be set of all one-point subsets of $X$, then it is a
    basis for the discrete topology on $X$.
  \end{remark}
  \begin{remark}
    So my understanding of a basis of a topology is like the generator
    of the topology that satisfy the existence of intersection. So
    first of all we have to note that $\B\subset\T$.

    Using the process from the book, we can verify it. Take $J$ to be
    an indexed family of $\B$.

    It's easy to show that $\bigcup_{\alpha\in J} B_\alpha$ is in $\T$
    if $\forall \alpha\in J : B_\alpha\in\T$.

    It's easier to to show that $\bigcap_{\alpha\in J} B_\alpha$ is in
    $\T$ if $\forall \alpha\in J: B_\alpha\in\T$ since it's specified
    in the criteria for $\B$ to be a basis. (Actually basis does more
    than finite intersection. We can prove using induction that the
    intersection of any countable number of basis is in $\T$.

    Also, $\emptyset\in\T$ will be vacuously true no matter what $\B$
    we pick.
  \end{remark}
  \begin{theorem}
    A more visual-able theorem. $\T$ is the all unions of elements in
    $\bigcup_{B\in\B} B$.
  \end{theorem}
  \begin{proof}
    Given $B=\bigcup_{\alpha\in K} B_\alpha$, since $B_\alpha\in\T$
    and for any $x,y\in\T$, $x\cup y\in\T$. Thus $B\in\T$. Conversely,
    given $U\in\T$, choose for each $x\in U$, $\ex B_x\in\B$ and
    $U=\bigcup_{x\in U} B_x$. Therefore $U$ is some union of elements
    in $\B$.
  \end{proof}
  \begin{theorem}
    Let $X$ be a topological space. $\mathcal{C}\subset \T$ such that
    for each open set $U\in\T$ and each $x\in U$, there is an element
    $C\in\mathcal{C}$ such that $x\in C\subset U$. Then $\mathcal{C}$
    is a basis for $X$.
  \end{theorem}
  \begin{theorem}
    Let $\B$ and $\B'$ be bases for $\T$ and $\T'$, respectively. Then
    the following are equivalent:
    \begin{enumerate}
    \item $\T'$ is finer than $\T$
    \item
      $\forall x\in X, B\in\B : x\in B \RA \exists B'\in\B' : x\in B'
      \subset B$.
    \end{enumerate}
  \end{theorem}
  \begin{remark}
    Think of a finer topology to be a set with more elements, while
    inclusively. Then it works just like the defintion of
    ``subsets''. While we are now not talking about the topologies but
    the bases, so we only care about the elements in the bases.

    The book uses the concept of a gravel. So the pebbles forms a
    basis of a topology. When they get smashed into dust, they form
    the basis of a new topology, while finer. And the dust particles
    was contained inside a pebble, as says the criterion.
  \end{remark}
  \begin{example}
    One can be demonstrated that topology on $\RR$ generated by open
    circles is the same topology as generated by rectangles. Below
    diagram shows this:

    \begin{figure}[H]
      \centering
      \begin{tikzpicture}
        \node (x) [point,label={[name=xl] below:x}] {};
        \node (B) [fit=(x)(xl),circle,draw,label={[name=Bl] right:B}] {};
        \node (Bp) [fit=(B)(Bl),rectangle,draw,label=below:B'] {};

        \node (x2) [point,right of=Bp,xshift=60pt,label={[name=x2l]
          below:x}] {};
        \node (B2) [fit=(x2)(x2l),rectangle,draw,label={[name=B2l]
          right:B}] {};
        \node (B2p) [fit=(B2)(B2l),circle,draw,label=below:B'] {};
      \end{tikzpicture}
    \end{figure}

    ``Since for each point $x$ and each basis element $B\in\B$
    containing x, there is a element $B'\in\B'$ such that $x\in
    B'\subset B$.'' -- This applies to $B$ to be the circluar basis
    and the rectangular basis. Thus we conclude that $T\subset T'$ and
    $T'\subset T$, thus they are equivalent.
  \end{example}

  \begin{definition}\defn{Standard topology}, \defn{Lower limit
      topology} and \defn{K topology}
    The \emph{standard topology} has the basis of all open intervals in the
    real line. If $\B=\set{[a,b)\mid a,b\in\RR }$, the topology
    generated by $B$ is then called the \emph{lower limit topology} on
    $\RR$, denoted $\RR_l$. Let $K$ denote the set of all numbers of
    the form $1/n$, $n\in\ZZ^+$, and let $B'$  be the set of all open
    intervals in form of $(a,b)-K$, is called \emph{K-topology},
    denoted as $\RR_k$.
  \end{definition}
  \begin{proposition}
    $\RR_l$ and $\RR_k$ are strictly finer than the standard $\RR$,
    while are not comparable with one another.
  \end{proposition}

  \begin{definition}\defn{Subbasis}
    A subbasis $S$ for a topology on X is a set of subsets of $X$
    whose union equals to X.
  \end{definition}
  \begin{remark}
    Unlike a subset/subgroup/subspace, a subbasis is not as its name
    suggests to be a sbuset of some other basis. Subbasis and basis
    are two different way to generate a topology. A basis includes all
    intersection of two basis elements, while a subbasis doesn't have
    to. So to generate a topology using a basis, we take all
    union. While to generate a topology using a subbasis, we take all
    unions and intersection of subbasis elements. (ref. {Eric Auld on
      Math
      Stack-Exchange\link{https://math.stackexchange.com/a/449577/120022}},
    and Wikipedia\link{https://en.wikipedia.org/wiki/Subbase}.
  \end{remark}
  \begin{example}ref. \link{http://math.stackexchange.com/a/449593/120022
    }\\
    $\T=\set{\emptyset, \set{0}, \set{0,1}, \set{0,2}, \set{0,1,2}}$
    \\
    $\B=\set{\set{0},\set{0,1},\set{0,2}}$ \\
    $\mathcal{S}=\set{\set{0,1},\set{0,2}}$
  \end{example}
  \begin{example}
    (ref. \link{http://mathworld.wolfram.com/Subbasis.html}) For standard topology on $\RR$, $\T$ is all open intervals (and
    their unions) on $\RR$. Then, \\
    $\B=\set{(a,b)\mid a,b\in\RR,\,a\leq b}$, and \\
    $\mathcal{S}=\set{(-\infty,a)\mid
      a\in\RR}\cup\set{(b,\infty)\mid b\in \RR}$ \\

  \end{example}

  \begin{definition}\defn{Order topology}
    Assuming we knowing about all open/closed/half-open interval
    concepts, \emph{over a, possibly not continuous, set}.

    Let $X$ be a set with linear order relation with $<$. Assume
    $|X|>1$. Let $\B$ be defined as a set of all subsets of $X$ of the
    following types:
    \begin{enumerate}
    \item All valid $(a,b)$
    \item For $a_0$ to be the minimal element, $[a_0,b)$ (if any)
    \item For $b_0$ to be the maximal element, $(a,b_0]$ (if any)
    \end{enumerate}

    Then $B$ is a basis for a topology on $X$, called \emph{order
      topology}.
  \end{definition}
  \begin{example}
    $\RR$, $\RR\times\RR$ in dict order, $\ZZ^+$
  \end{example}
  \begin{definition}\defn{Ray}, \defn{Open ray}, \defn{Closed ray}
    \begin{align*}
      (a,+\infty)&=\set{x\mid x>a} \\
      (-\infty,a)&=\set{x\mid x<a} \\
      [a,+\infty)&=\set{x\mid x\geq a} \\
      (-\infty,a]&=\set{x\mid x\leq a}
    \end{align*}
  \end{definition}

  \begin{definition}\defn{Product topology}
    Product topology on topology spaces $X$ and $Y$, denoted as
    $X\times Y$ is the topology with the basis $\B$ who elements are
    in form of $U\times V$ where $U\in\T_X, V\in\T_Y$. (Obviously and
    omitted by book the underlying set is just $X\times Y$)
  \end{definition}
  \begin{example}
    open sets on $\RR\times \RR=\RR^2$.
  \end{example}
  \begin{theorem}
    If $\B$ is a basis for topology  of $X$ and $\mathcal{C}$ is a
    basis for the topology of $Y$, then the collection
    $\mathcal{D}=\set{B\times C\mid B\in \mathcal{B}\land
      C\in\mathcal{C}}$
    is a basis for $X\times Y$.
  \end{theorem}
  \begin{definition}\defn{Projection}
    $$\pi_1:X\times Y\to X,\quad\pi_2:X\times Y\to Y$$
    Projection functions are \emph{onto}.
  \end{definition}

  \begin{definition}\defn{Subspace} and \defn{Subspace topology}
    For $X$ be topological space and $\T$ be topology on $X$. If
    $Y\subset X$, then define
    $$\T_Y=\set{Y\intersect U\mid U\in\T}$$ \emph{subspace topology}
    of $\T$. $\T_Y$ is a topology.
  \end{definition}
  \begin{remark}
    Does $Y$ have to an open set in $X$ (or $\T$)? No.
    \begin{itemize}
    \item remember $X$ does not have any structure, it is $\T$ who
      gives the structure to $X$;
    \item it is not necessary for $Y\in\T$, as I will prove $\T_Y$ is
      a topology anyway below.
    \end{itemize}
  \end{remark}
  \begin{proof}
    $\emptyset=Y\intersect\emptyset\in\T_Y$, $Y=Y\intersect X\in\T_Y$,
    for $U_1,U_2\in\T_Y$,
    $U_1\intersect U_2=V_1\intersect Y\intersect V_2\intersect
    Y=(V_1\intersect V_2)\intersect Y\in \T_Y$ where $V_1,V_2\in\T$,
    for $U_1,U_2\in\T_Y$, $U_1\union U_2=(V_1\intersect Y)\union
    (V_2\intersect Y)=(V_1\union V_2)\intersect Y\in\T_Y$.

    This proof is not correct. A topology requires arbitrary union \ie
    $\Union_{\alpha\in J}$ and finite intersection
    $U_1\intersect\dots\intersect U_n$. But the same rule applies.
  \end{proof}
  \begin{theorem}
    Let $\B$ be basis for a topology of $X$, then
    $\B_Y=\set{B\intersect Y\mid B\in\B}$
    is a basis for the subspace topology on $Y$.
  \end{theorem}
  \begin{theorem}
    If $Y$ is subspace of $X$. If $U\in\T_Y$, and $Y\in\T$, then
    $U\in\T$.
  \end{theorem}
  This is just what I concerned about above. Pretty straightforward
  result.
  \begin{theorem}
    $A\subset X$ and $B\subset Y$, then $A\times B$ is the same
    topology $A\times B$ that inherits as subspace of $X\times Y$.
  \end{theorem}
  \begin{definition}\defn{Convex}
    Given an ordered set $X$, $Y\subset X$. Say $Y$ is \emph{convex in
    } $X$ if for any $a<_Yb$, $(a,b)\in X\RA (a,b)\in Y$.
  \end{definition}
  \begin{example} Here are some examples of $Y$ convex in $X$.
    \begin{itemize}
    \item $X=\RR$, $Y=[0,1]$
    \item $X=\RR$, $Y=(0,1)$
    \item $X=[0,2]\union[3,5]$, $Y=[1,2]\union[3,4]$
    \item $X=[0,2]\union[3,5]$, $Y=[4,5]$
    \end{itemize}
    Here are some counterexamples:
    \begin{itemize}
    \item $X=\RR$, $Y=[0,1)\union(1,2]$
    \item $X=\RR$, $Y=[0,1]\union[2,3]$
    \item $X=[0,2]\union[3,5]$, $Y=[1,2]\union[4,5]$
    \end{itemize}
  \end{example}
  \begin{theorem}
    Let $X$ be ordered set in order topology; let $Y\subset X$ be
    convex in $X$. Then the order topology on $Y$ is the same as the
    topology $Y$ inherits as a subspace of $X$.
  \end{theorem}
  \begin{definition}\defn{Closed set}
    A subset $A$ of a topological space $X$ is said to be
    \emph{closed} if the set $X-A$ is open.
  \end{definition}
  \begin{remark}
    A subset $A\in X$ is closed DOES NOT mean itself is not open. It
    means its complement is open. An example: trivial topology on
    $\set{a,b}$ includes $\set{\emptyset, \set{a,b}}$. Then $\set{a}$
    is neither closed or open. In the discrete topology, every point
    is both closed and open.
  \end{remark}
  \begin{theorem} Let $X$ be a topological space, the following
    conditions hold:
    \begin{enumerate}
    \item $\emptyset$ and $X$ are closed.
    \item Arbitrary intersections of closed sets are closed.
    \item Finite unions of closed sets are closed.
    \end{enumerate}
  \end{theorem}
  \begin{theorem}
    Let $Y$ be subspace of $X$. Then a set $A$ is closed in $Y$ iff it
    squals the intersection of a closed set of $X$ in $Y$.
  \end{theorem}

  \begin{definition}\defn{Closure} and \defn{interior} of a Set
    Given $A\subset X$, the \emph{interior} of $A$ is definted as the union
    of all open sets contained in $A$, and the \emph{closure} of $A$
    is defined as the intersection of all closed sets containing $A$.

    Interior of $A$ is denoted as $\Int a$, and closure of $A$ is
    denoted as $\bar{A}$. I will use $\interior{A}$ and $\closure{A}$
    to denote these two in this note.
  \end{definition}
  \begin{remark}
    $\interior{A}\subset A\subset \closure{A}$
  \end{remark}
  \begin{remark}
    If $A$ is open, $A=\interior{A}$; if $A$ is closed,
    $A=\closure{A}$.
  \end{remark}
  \begin{theorem}
    Let $Y$ be a subspace of $X$. Let $A$ be a subset of $Y$. Let
    $\closure{A}$ be closure of $A$ in $X$. Then the closure of $A$ in
    $Y$ equals $\closure{A}\intersect Y$.
  \end{theorem}
  \begin{remark}
    When we talk about $Y$ subspace of $X$ and $A\subset Y$. We
    reserve notation $\closure{A}$ to denote the closure of $A$ in
    $X$.
  \end{remark}
  \begin{definition}\defn{Intersects}
    $A$ intersects $B$ iff $A\intersect B\not=\emptyset$.
  \end{definition}
  \begin{theorem} Let $A$ be a subset of topology space $X$.
    \begin{enumerate}
    \item Then $x\in\closure{A}$ iff every open set $U$ containing $x$
      intersects $A$.
    \item Suppose topology of $X$ is given by a basis $\B$, then $x\in
      A$ iff $\forall B\in\B:x\in B\RA B\intersect A\not=\emptyset$.
    \end{enumerate}
  \end{theorem}
  \begin{definition}\defn{Neighborhood}
    We shorten the statement ``$U$ is an open set containing $x$'' to
    the phrase ``$U$ is a neighborhood of $x$''.
  \end{definition}
  \begin{definition}\defn{Limit point}
    If $A$ a subset of the topological space $X$ and if $x$ is a point
    of $X$, we say that $x$ is a \emph{limit point} (\aka
    ``\emph{cluster point}'') of $A$ if every neighborhood of $x$
    intersects $A$ in some point other than $x$ itself.
  \end{definition}
  \begin{example}
    Consider standard topology on $\RR$. If $A=(0,1]$, then $0$ is a
    limit point. So is $\frac{1}{2}$. Any $x\in[0,1]$ is a limit
    point, but no other points.
  \end{example}
  \begin{example}
    On $\RR$, if $B=\set{1/n\mid n\in\ZZ^+}$, then $0$ is the only
    limit point of $B$.
  \end{example}
  \begin{theorem}
    Let $A$ be a subset of topological space $X$. Let $A'$ be the set
    of all limit points of $A$. Then
    $\closure{A}=A\union A'$
  \end{theorem}
  \begin{corollary}
    A subset of a topoloical space is closed iff it contains all its
    limit points.
  \end{corollary}
  \begin{definition}\defn{Converge}
    The convergency property of $x\in\RR$ is generalized in topology
    as:
    One say that a sequence $x_1,x_2,\ldots$ of points of the spacec
    $X$ converges to the point $x$ of $X$ if that corresponding to
    each neighborhood $U$ of $x$, there is a positive integer $N$ such
    that $x_n\in U$ for all $n\geq N$.
  \end{definition}
  \begin{definition}\defn{Hausdorff space}
    A topological space $X$ is called a \emph{Hausdorff space} if for
    each pair $x_1$, $x_2$ of distinct points of $X$, there exist
    neighborhoods $U_1$, and $U_2$ of $x_1$ and $x_2$, respectively,
    that are disjoint.
  \end{definition}
  \begin{theorem}
    Every finite point set in a Hausdorff space $X$ is closed.
  \end{theorem}
  \begin{proof}
    We show any one point set $\set{x_0}$ is closed. If $x$ is a point
    in $X$ different from $x_0$, then $x$ and $x_0$ have disjoint
    neighborhoods $U$ and $V$, repsectively. Since $U$ does not
    intersect $\set{x_0}$, the oint $x$ cannot belong to the closure
    of the set $\set{x_0}$. Therefore the closure of the set
    $\set{x_0}$ is itself. Therefore it is closed.
  \end{proof}
  \begin{remark}
    This condition that finite point sets is closed is actually weaker
    than the Hausdorff condition. \eg $\RR$ in the finite complemenet
    topology is not a Hausdorff space but every finite point set is
    closed. This condition is named as \defn{T$_1$ axiom}.
  \end{remark}
  \begin{theorem}
    Let $X$ be a space satisfying $T_1$ axiom; let $A$ be subset of
    $X$. Then the point $x$ is a limit point of $A$ iff every
    neighborhood of $x$ contains infinitely many points of $A$.
  \end{theorem}
  \begin{theorem}
    $X$ Hausdorff space, a sequence of points in $X$ converges to at
    most one point of $X$. \defn{Limit} is called for that point
    of the sequence.
  \end{theorem}
  \begin{theorem}
    Simply ordered set is a Hausdorff space in order topology.
    Product of two Hausdorff space is a Hausdorff space.
    A subspace of a Hausdorff space is a Hausdorff space.
  \end{theorem}
  \begin{definition}\defn{Continuous function}
    Let $X$ and $Y$ be topological spaces. A function $f:X\to Y$ is
    said continuous if for every open subset $V$ of $Y$, the set
    $f^{-1}(V)$ is open subset of $X$.
  \end{definition}
  \begin{remark}
    Remember in the defintion, it does not say $f$ maps all open sets
    in $X$ to open sets in $Y$. It says $f^{-1}$ maps all open sets in
    $Y$ to open sets in $X$.
  \end{remark}
  \begin{theorem}
    To show a function is continuous, one only need to show it's
    inverse image for each subbasis element is open.
  \end{theorem}
  \begin{remark}
    I've proven this with Prof. Erickson on his class. The final
    result come out to be the inversed image will become the basis.
    This shows one usefulness of subbasis.
  \end{remark}
  \begin{example}
    Analysis analogy: let's study real function $f : \RR \to \RR$.

    In analysis, we define continuity of $f$ via the $\epsilon-\delta$
    definition. Our definition will imply the $\epsilon-\delta$
    definition as follows.

    Definition for continuity for real valued function: \\
    $\forall x_0\in \operatorname{Dom}(f):\forall \epsilon>0:\exists \delta: \forall
    x\in(x_0-\delta,x_0+\delta):
    f(x)\in(f(x_0)-\epsilon,f(x_0)+\epsilon)$

    \begin{table}[H]
      \centering
      \begin{tabular}[H]{ll}
        analysis language & topology language \\
        \hline
        $\RR$ & $\RR$ with standard topology \\
        $f : X \to Y$ ($X,Y: \text{Set}$)
                          & $f : X \to Y$ ($X,Y: \text{Topological space}$) \\
        open interval & open set \\
        $\delta-$neighborhood of $x_0$
                          & open set in $X$ containing $x_0$ \\
        $\epsilon-$neighborhood of $f(x_0)$
                          & open set in $Y$ containing $f(x_0)$
      \end{tabular}
    \end{table}

    Restatement of above definition for continuity in Topology language: \\
    $\forall x_0\in X:\forall U_Y\in\T_Y:\exists U_X\in\T_X\land
    x_0\in U_X:\forall x\in U_X:f(x_0)\in U_Y$ \\
    Then we will discover that we can actually go more general: \\
    $\forall U_Y\in\T_Y:\exists U_X\in\T_X :\forall x\in U_X:f(x)\in
    U_Y$ \\
    And even more general: \\
    $\forall U_Y\in\T_Y:\exists U_X\in\T_X:f(U_X)=U_Y$ \\
    Then it comes to be continuity function definition in topology: \\
    $\forall U_Y\in\T_Y:f^{-1}(U_Y)\in\T_X$
  \end{example}
  \begin{theorem}
    Let $X$ and $Y$ be topological spaces; let $f:X\to Y$. Then the
    following are equivalent:
    \begin{enumerate}
    \item $f$ is cont.
    \item For every $A\subset X$,
      $f(\closure{A})\subset\closure{f(A)}$
    \item For every closed set $B\subset Y$, $f^{-1}(B)$ is closed in
      $X$.
    \item For each $x\in X$ and each neighborhood $V$ of $f(x)$, there
      is a neighborhood $U$ of $x$ such that $f(U)\subset V$.
    \end{enumerate}
  \end{theorem}
  \begin{definition}\defn{Homeomorphism}
    Let $f:X\to Y$ be bijection. If both $f$ and $f^{-1}$ are
    continuous, then $f$ is called a \emph{homeomorphism}.
  \end{definition}
  \begin{remark}
    One point to be noticed. This definition requires $f$ and $f^{-1}$
    both be continuous. You may ask, if $f$ is bijective and $f$ is
    continuous, doesn't it follows automatically that $f^{-1}$ is
    continuous? Of course not. Here's a counter-example.

    Take $f : \RR \to \RR$ with $f(x)=x$, the first is the discrete
    topology on $\RR$ and the second is the standard topology
    $\RR$. Then $f$ is obviously bijective, and $f$ is open since
    $f^{-1}(U)$ is always open, so $f$ is continuous. But the converse
    isn't true.
  \end{remark}
  \begin{definition}
    An alternative definition for homeomorphism is a bijective
    function such that $f(U)$ is open iff $U$ is open.
  \end{definition}
  \begin{definition}\defn{Imbedding}, \defn{Embedding}
    An injective continuous map is called an \emph{imbedding} (\aka
    \emph{embedding}.
  \end{definition}
  \begin{definition}\defn{Unit circle}, \defn{S$^1$}
    $S^{1}=\set{x\times y\mid x^2+y^2=1}$
  \end{definition}
  \begin{theorem}Constructing continuous map
    \begin{enumerate}
    \item Constant map.
    \item Inclusion map.
    \item Composition of continuous map.
    \item Restricting domain. $f:X\to Y$ is cont $\RA$ $f|A:A\to Y$
      where $A\subset X$.
    \item Restricting the range.
    \item Local formulation of continuity. If $X$ can be written as
      union of open sets $U_\alpha$, and $f|U_\alpha$ is cont. Then
      $f:X\to Y$ is cont.
    \end{enumerate}
  \end{theorem}
  \begin{theorem} (\textbf{The }\defn{Pasting Lemma}).
    Let $X= A\union B$ where $A$ and $B$ are closed in $X$. Let
    $f:A\to Y$ and $g:B\to Y$ cont. If $f(x)=g(x)$ for every $x\in
    A\intersect B$. Then $f$ and $g$ combine to give a cont func
    $h:X\to Y$.
  \end{theorem}
  \begin{theorem} (\defn{Maps into products})
    Let $f:A\to X\times Y$
    given by $f(a) = (f_1(a), f_2(a))$ then $f$ is cont iff
    $f_1:A\to X$ and $f_2:A\to Y$ are cont. The maps $f_1$ and $f_2$
    are called the \defn{Coordinate functions} of $f$. (sort of like
    projection map).
  \end{theorem}
  \begin{definition}\defn{Box topology}
    For $\prod{X_i}$, either finite or infinite, if we take the
    cartesian product of the basis/open sets, correspondingly, we get
    a basis for the new topology. This is called the box
    topology. (Just as what we defined above for product topology)
  \end{definition}

  \begin{definition}\defn{Metric}
    A metric on a set $X$ is a function $d:X\times X\to\RR$ having the
    following properties:
    \begin{enumerate}
    \item $d(x,y)\geq 0$ for all $x,y\in X$; equal only when $x=y$
    \item $d(x,y)=d(y,x)$ for all $x,y$
    \item $d(x,y)+d(y,z)\geq d(x,z)$ for all $x,y,z$ (\defn{Triangle inequality})
    \end{enumerate}
  \end{definition}
  \begin{definition}\defn{Epsilon ball}
    $B_d(x,\epsilon)=\set{y\mid d(x,y)<\epsilon}$
  \end{definition}
  \begin{definition}\defn{Metric topology}
    If $d$ is a metrix on $X$. The collection of all $\epsilon$-balls
    $B_d(x,\epsilon)$ for all $x\in X$ and all $\epsilon>0$ as basis
    of a topology on $X$. This topology is called the metric topology
    inducted by $d$.
  \end{definition}
  \begin{example}
    $\RR$ under standard topolgoy is a metric topology induced by
    the metric ``absolute value of difference function''.
  \end{example}
  \begin{example}
    The discrete topology for a set $X$ is given by the metric
    topology induced by $d$ defined as:
    $$\begin{cases}d(x,y)=1 & \text{if }x\not=y \\ d(x,y)=0 &
      \text{if }x=y\end{cases}$$
  \end{example}
  \begin{definition}\defn{Metrizable}, \defn{Metric space}
    trivial.
  \end{definition}
  \begin{definition}\defn{Bounded}, \defn{Diameter}
    Bounded if $d(a_1,a_2)\leq M$. \\
    $\operatorname{diam} A = \sup\set{d(a_1,a_2)\mid a_1,a_2\in A}$
    called diameter.
  \end{definition}
  \begin{definition}\defn{Standard bounded metric}
    Let $X$ be a metric space with metric $d$. Define
    $\overline{d}:X\times X\to\RR$ by the equation.
    $$\overline{d}(x,y)=\operatorname{min}\set{d(x,y),1}$$ then
    $\bar{d}$ is a metric that induces the same topology as $d$.
  \end{definition}
  \begin{definition}\defn{Norm}, \defn{Euclidean metric}, \defn{Square metric}
    trivial for norm. trivial for euclidean metric. \\
    square metric
    $\rho$ is defined by $\rho(x,y)=\max\set{|x_1-y_1|,\ldots,|x_n-y_n|}$
  \end{definition}
  \begin{theorem}
    Let $d$ and $d'$ be two metrics on the set $X$; let $\T$ and $\T'$
    be the topology they induce, respectively. Then $\T'$ is finer
    than $\T$ iff for each $x$ in $X$ and each $\ep >0$, there is a
    $\delta>0$ such that $B_{d'}(x,\delta)\subset B_d(x,\ep)$.
  \end{theorem}
  \begin{remark}
    This result is obvious since finer basis determines finer topology.
  \end{remark}
  \begin{theorem}
    The topologies on $\RR^n$ induced by the euclidean metric $d$ and
    the square metric $\rho$ are the same as the product topology on
    $\RR^n$.
  \end{theorem}
  \begin{definition}\defn{Uniform metric}, \defn{Uniform topology}
    Given an index set $J$ and given points
    $\mathbf{x}=(x_\alpha)_{\alpha\in J}$ and
    $\mathbf{y}=(y_\alpha)_{\alpha\in J}$ of $\RR^J$, let us define a
    metric $\bar{\rho}$ on $\RR^J$ by the equation
    $$\bar\rho(\mathbf{x},\mathbf{y})=\sup\set{\bar{d}(x_\alpha,y_\alpha)\mid
      \alpha\in J}$$

    This is called the \emph{uniform metric} on $\RR^J$ and the
    topology it induces is called the \emph{uniform topology}.
  \end{definition}
  \begin{theorem}
    The uniform topology on $\RR^J$ is finer than the product topology
    and coarser than the box topology; these topologies are all
    different if $J$ is infinite.
  \end{theorem}
  \begin{theorem}
    Let $\bar{d}(a,b)=\min\set{|a-b|,1}$ be the standard bounded
    metric on $\RR$. If $x$ and $y$ are two points of $\RR^\omega$,
    define
    $$D(\mathbf{x},\mathbf{y})=\sup\set{\frac{\bar{d}(x_i,y_i)}{i}}$$
    to be a metric on $\RR^\omega$. Then the topology induced by $D$
    is the product topology.
  \end{theorem}
  \begin{remark}
    needs more explanation on these two.
  \end{remark}
  \begin{theorem}
    Let $f:X\to Y$; let $X$ and $Y$ be metrizable with metrics $d_X$
    and $d_Y$, respectively. Then continuity of $f$ is equivalent to
    the requirement that given $x\in X$ and given $\ep>0$, there
    exists $\delta>0$ such that
    $$d_X(x,y)<\delta\RA d_Y(f(x),f(y))<\ep$$
  \end{theorem}
  \begin{theorem}\defn{The sequence lemma}
    Let $X$ be topological space, let $A\subset X$. If there is a
    sequence of points of $A$ converging to $x$, then $x\in \bar{A}$;
    the converse holds if $X$ is metrizable.
  \end{theorem}
  \begin{theorem} Let $f:X\to Y$. If the function $f$ is continuous,
    then for every convergent sequence $x_n\to x$ in $X$, the sequence
    $f(x_n)$ converges to $f(x)$. The converse holds if $X$ is
    metrizable.
  \end{theorem}
  \begin{theorem}
    If $X$ is topological space and $f,g:X\to \RR$ are cont. Then
    $f+g,f-g,fg$ are cont and $f/g$ are cont for $g\not=0$.
  \end{theorem}
  \begin{definition}\defn{Converge uniformly}
    Let $f_n:X\to Y$ be a sequence of functions from the set $X$ to
    the metric space $Y$. Let $d$ be the metric for $Y$. We say that
    the sequence $(f_n)$ \emph{converges uniformly} to the function
    $f:X\to Y$ if given $\ep>0$, there exists an integer $N$ such that
    $$d(f_n(x),f(x))<\ep$$ for all $n>N$ and all $x$ in $X$.
  \end{definition}
  \begin{theorem}\defn{Uniform limit theorem} Let $f_n:X\to Y$ be a
    sequence of functions from topological space $X$ to metric space
    $Y$. If $(f_n)$ converges uniformly to $f$ then $f$ is continuous.
  \end{theorem}
  \begin{definition}\defn{Quotient map}
    Let $X$ and $Y$ be topological spaces; let $p:X\to Y$ be a
    surjective map. The map $p$ is said to be a quotient map if a
    subset $U$ of $Y$ is open iff $p^{-1}(U)$ is open in $X$.
  \end{definition}
  \begin{remark}
    It sounds like definition for continuity. The difference is in
    continuity we specify the inverse mapping from open sets to open
    sets. In this one, we have to inversely map open sets to open
    sets, also non-open set to non-open set. This condition is
    sometimes called \defn{Strong continuity}.

    Equivalently we can define quotient maps as $A\subset Y$ closed
    iff $p^{-1}(A)$ closed in $X$. It follows that for all
    $B\subset Y$:

    $$f^{-1}(Y-B)=X-f^{-1}(B)$$
  \end{remark}
  \begin{definition}\defn{Saturated} A set $C\subset X$ is saturated (with
    respect to the surjective map $p:X\to Y$) if $C$ contains every
    set $p^{-1}({y})$ that $C$ intersects. (symbolically, $C$ is
    saturated if $p\inv({y})$

    A saturated set is just a set $C\subset X$ that is an intersection
    of open sets of
    $X$. \link{https://en.wikipedia.org/wiki/Saturated_set}
  \end{definition}
  \begin{remark}
    The definition on the book is not that clear to me. Wikipedia
    explains it in an intuitive way (The second line in above
    definition).
  \end{remark}
  \begin{remark}
    To really understand what a saturated set is, we have to do it
    under a quotient map $q : X \to p(X)$. A quotient map $p$ define
    an equiv relation on $X$. Say $A\in p(X)$ consists of points
    $\set{a,b}\in X$, then $\set{a,b}$ is a saturated set whereas
    $\set{a}$ or $\set{b}$ isn't. Every saturated set $A$ can be
    expressed as $q^{-1}q(A)$.
    \link{http://math.stackexchange.com/a/1173764/120022}
  \end{remark}
  \begin{definition}\defn{Open map}, \defn{Closed map}
    An open map is a map that maps open sets to open sets.
    An closed map is a map that maps closed sets to closed sets.
  \end{definition}
  \begin{remark}
    Continuous map requires $f^{-1}(U)$ open if $U$ open. Open map
    requires $f(U)$ open if $U$ open.
  \end{remark}
  \begin{example}
    Let $X=Y=\set{a,b}, \T_X=\T_Y=\set{\emptyset, \set{a}, \set{a,b}}$
    \begin{enumerate}
    \item Open map not closed: $f~x = a$
    \item Closed map not open: $f~x = b$
    \end{enumerate}
    Above two maps are neither continuous
  \end{example}
  \begin{definition}\defn{Quotient map}
    A quotient map is a continuous surjective map that is either open
    or closed or both.
  \end{definition}
  \begin{definition}\defn{Quotient topology}
    Let $p : X \to A$ be a quotient map from topological space $X$ to
    set $A$. Then $A$ is called a quotient topology.
  \end{definition}
  \begin{definition}\defn{Quotient space}
    Let $X$ be a topological space, and let $X^*$ be a partition of
    $X$ into disjoint subsets whose union is $X$. Let $p : X\to X^*$
    be the surjective map carries each point of $X$ to the
    corresponding partition in $X^*$ that contains it. $X^*$ is
    called a \emph{quotient space} of $X$.
  \end{definition}
  \begin{remark}
    Quotient space give rise to a quotient topology by identifying
    points into equivalence relations. Notice that usually only very
    few distinct points of $X$ will be put into a common partition and
    the rest points in $X$ will usually belong to a partition with
    itself only. $X^*$ is also called \defn{Identification space} or a
    \defn{Decompostion space} of $X$.
  \end{remark}
  \begin{theorem}
    Let $p:X\to Y$ be quotient map; Let $A\subset X$ be saturated with
    respect to $p$; then let $q: A\to p(A)$ be the map obtained by
    restricting $p$.
    \begin{enumerate}
    \item If $A$ is either open or closed in $X$, then $q$ is a
      quotient map.
    \item If $p$ is either an open map or a closed map, then $q$ is a
      quotient map.
    \end{enumerate}
  \end{theorem}
  \begin{theorem}
    Composition of quotient maps is quotient map. Can be shown by
    $p^{-1}(q^{-1}(U)) = (q\circ p)^{-1}(U)$.
  \end{theorem}
\end{chapter}

\begin{chapter}{Connectedness and Compactness}
  \begin{remark}
    Connectedness and compactness gives a generalization of IVT, MVT
    and UCT(Uniform continuity theorem). IVT depends on the property
    of connectedness on the space. MVT and UCT depends on compactness.
  \end{remark}
  \begin{definition}\defn{Separation}
    A separation of $X$ is a pair $U,V$ of \underline{disjoint},
    \underline{nonempty}, \underline{open} subsets of $X$ whose union
    is $X$.
  \end{definition}
  \begin{definition}\defn{Connected}
    A space is connected if there is no separation.
  \end{definition}
  \begin{definition}
    Another definition: A space is connected iff the only subsets that
    are both open and closed are $X$ and $\emptyset$.
  \end{definition}
  \begin{proof}
    (\RA) If there is a separtion $U,V$, $U,V\in\T$, then $U=X-V$,
    $V=X-U$ are both open and closed. (\LA) If there is a clopen set
    $U$, $U\not=X, U\not=\emptyset$. Then $X-U$ and $U$ form a
    separation of $X$.
  \end{proof}

  \begin{theorem}
    If $Y$ subspace $X$, a separation of $Y$ is a pair of disjoint
    nonempty sets $A$ and $B$ whose union is $Y$. Neither of which
    contains a limit point of the other. The space $Y$ is connected if
    there exists no separtion of $Y$.
  \end{theorem}
  \begin{example}
    \begin{enumerate}
    \item A two point set in the indiscrete topology (contains only
      $\emptyset$ and $X$) is connected since there is no separtion.
    \item $[-1,1] - \set{0}$ is disconnected. (limit point of both: $0$)
    \item $\QQ$ is disconnected. Any one point set and its
      complement are clopen.
    \item $\set{x\times 0}\union \set{x \times 1/x\mid x>0}$ is
      disconnected.
    \end{enumerate}
  \end{example}

  \begin{theorem}
    If $C$ and $D$ separates $X$, and if $Y$ is a connected subspace
    of $X$, then $Y$ lies entirely in $C$ or $D$.
  \end{theorem}

  \begin{theorem}
    The union of a collection of connected subspaces of $X$ that have
    a point in common is connected.
  \end{theorem}

  \begin{theorem}
    Let $A$ be connected subspace of $X$. If $A\subset B\subset
    \closure{A}$, then $B$ is connected.

    Said differently, if $B$ is formed by adjoining a connected
    subspace $A$ and all its limit points, then $B$ is connected.
  \end{theorem}

  \begin{theorem}
    Continuous map preserves connectedness. (Very important result for
    this chapter)
  \end{theorem}
  \begin{theorem}
    Finite cartesian product of connected space is connected.
  \end{theorem}

  \begin{definition}\defn{Linear continuum}
    A simply ordered set $L$, with $|L|>1$ is a linear continuum if
    all following hold:
    \begin{itemize}
    \item $L$ has the LUB property (completeness?)
    \item For every $x<y$, exists some $z$ st $x<z<y$
    \end{itemize}
  \end{definition}

  \begin{theorem}
    If $L$ is a linear continuum in the order topology, then $L$ is
    connected. So are the intervals and rays in $L$.
  \end{theorem}

  \begin{theorem}\defn{Intermediate value theorem},\defn{IVT}
    If $f: X \to Y$ is a coontinuous map, where $X$ is connected space
    and $Y$ is an ordered set in the order topology. If $a$ and $b$
    are two points of $X$, and $r$ is a point in $Y$ lies between
    $f(a)$ and $f(b)$, then there exists some point $c$ of $X$ such
    that $f(c)=r$.
  \end{theorem}
  \begin{remark}
    Note that $X$ is not necessily defined to be an ordered set. We
    have no guarantee so far for where that $c$ will be on $X$.

    Example: Take $X=\CC$, $Y=\RR$,
    $f(t)=\set{(\sin(t), \cos(t), t) \mid 0 < t < 2\pi}$, the spiral
    line.
  \end{remark}

  \begin{remark}

  \end{remark}
\end{chapter}


\printindex

\end{document}

%%% Local Variables:
%%% mode: latex
%%% TeX-master: t
%%% End:
