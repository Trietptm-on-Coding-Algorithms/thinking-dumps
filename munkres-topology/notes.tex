\documentclass{report}

\usepackage[utf8]{inputenc}

\usepackage{imakeidx}
\makeindex[columns=2,title=Alphabetical Index]

\usepackage{tikz}
\usetikzlibrary{fit,shapes}
\tikzstyle{point}=[draw,fill=black,circle,inner sep=0pt,minimum size=2pt]

\usepackage{amsmath}
\usepackage{amssymb}

\usepackage{hyperref}
\newcommand*{\link}[1]{\href{#1}{(\underline{link})}%
  \footnote{\url{#1}}}

\usepackage{xspace}
\usepackage{float}
\usepackage{enumitem}

\usepackage{amsthm}
\newtheorem*{remark}{Remark}

\theoremstyle{definition}
\newtheorem{theorem}{Theorem}[chapter]
\newtheorem{definition}[theorem]{Definition}
\newtheorem{example}[theorem]{Example}
\newtheorem{property}[theorem]{Property}

\newcommand{\defn}[1]{\textbf{#1}\label{#1}\index{#1}}
\newcommand{\set}[1]{\ensuremath\{#1\}}

\newcommand{\ie}{\textit{i.e.}\xspace}
\newcommand{\aka}{\textit{a.k.a.}\xspace}
\newcommand{\eg}{\textit{e.g.}\xspace}

\newcommand{\RA}{\ensuremath\Rightarrow}

\newcommand{\ZZ}{\mathbb{Z}}
\newcommand{\RR}{\mathbb{R}}


\newcommand{\ex}{\ensuremath\exists}


\begin{document}


\title{Notes on James R. Munkres' Topology (2E)}
\author{Shou}
\date{\today}

\maketitle
\tableofcontents

\setcounter{chapter}{-1}
\begin{chapter}{Structure and reading plans}
Ch 1-8 is the part I, mainly for common topology. The part II includes
ch 9-14, that depends on ch 1-4, is about algebraic topology.

My plan is to read through ch 1-4 very quickly, within a weekend, and
then I will start reading ch 9+ simultaneously with W.S.Massey's
Algebraic topology: An induction.

Finally I wish I could finish all ch 1-8 and also some parts after ch 9.
\end{chapter}

\begin{chapter}{Set theory and logic}
  \begin{definition}{\defn{Order relation}}
    rel $C$ on set $A$ is called \emph{order relation} if
    \begin{enumerate}
    \item comparability, \aka totality for all non-eq elements, \ie
      $\forall x,y\in A, x\not=y\RA x C y\lor y C x$
    \item non-refl, \ie $\forall x, \neg(x C x)$
    \item trans, \ie $\forall x C y\land y C z, x C z$
    \end{enumerate}
    (\aka \defn{Linear order}, \defn{Simple order})
  \end{definition}
  \begin{remark}
    This relation is not the same as Linear order on Wikipedia
    \link{https://en.wikipedia.org/wiki/Total_order}. This order is
    actually the strict version of the Total order on wikipedia, \ie
    has non-refl property.
  \end{remark}

  \begin{definition}{\defn{Open interval}}
    if $X$ is a set and $<$ is an order rel, and if $a<b$ we use
    notation $(a,b)$ to denote $\set{x\in X\mid a<x<b}$, called
    \emph{open interval}.

    If $(a,b) = \emptyset$, then $a$ is called \emph{immediate
      precessor} of $b$ and $b$ called \emph{immediate successor} of
    $a$.
  \end{definition}
  \begin{remark}
    It makes more sense on $X$ is a discrete set. Since if
    $(a,b)$ is an open interval in $\RR$, $(a,b) = \emptyset
    \RA a = b$ which makes no sense on $a$ as an immediate precessor
    of $b$.
  \end{remark}

  \begin{definition}{\defn{Order type}}
    if $A$ and $B$ are two sets with $<_A$ and $<_B$. We say that $A$
    and $B$ have same \emph{order type} if $\ex f : A \to B$ that
    preserves order, \ie $$a_1 <_A b_1 \RA f(a_1) <_B f(b_1)$$
  \end{definition}
  \begin{remark}
    It's just a generalization of monotone function.
  \end{remark}

  \begin{definition}{\defn{Dictionary order relation}}
    if $A$,$B$ are two sets with $(<_A,<_B)$, defn an order for
    $A\times B$ by defining $$a_1\times b_1 < a_2\times b_2$$
    if $a_1<_A a_2$, or if $a_1=a_2\;\land\;b_1<_B b_2$.
  \end{definition}

  \begin{definition}{\defn{LUB property}/\defn{GLB property}}
    For $A$ and $<_A$, we say $A$ has \emph{LUB property} if
    $$\forall A_0\subset A,A\not=\emptyset\land\ex \text{upper bound for }
    A_0 \RA \ex \text{lub$\{A_0\}$}\in A$$
  \end{definition}
  \begin{example}
    $A=(-1,1)$. \eg $X=\set{1-\frac{1}{n}\mid n\in\ZZ^+}$ does not have an
    upper bound, thus vacuously true. $\set{-\frac{1}{n}\mid
      n\in\ZZ^+}$ has upper bound of any number in $[0,1)\subset A$, and
    $\operatorname{lub}(X)=0\in (-1,1)$.
  \end{example}
  \begin{example}
    Counterexample. $A=(-1,0)\cup(0,1)$. $\set{-\frac{1}{n}\mid n\in\ZZ^+}$ has
    upper bound of any $(0,1)\subset A$, while
    $\operatorname{lub}(X)=0\not\in A$.
  \end{example}
  \begin{remark}
    The completeness property of $\RR$ as an axiom derives this property.
  \end{remark}

  \begin{property}{$\RR$ field}

    \textbf{Algebraic properties}
    \begin{enumerate}
    \item assoc: $(x+y)+z=x+(y+z); (xy)z=x(yz)$
    \item comm: $x+y=y+x; xy=yx$
    \item id: $\ex!0, x+0=x$; $\ex!1,x\not=0\RA x1=x$
    \item inv: $\forall x,\ex!y,x+y=0$; $\forall x\not=0,\ex!y,xy=1$
    \item distr: $x(y+z)=xy+xz$
    \end{enumerate}

    \textbf{Mixed algebraic and order property}
    \begin{enumerate}[resume]
    \item $x>y\RA x+z>y+z$; $x>y\land z>0\RA xz>yz$
    \end{enumerate}

    \textbf{Order properties}
    \begin{enumerate}[resume]
    \item $<$ has LUB property
    \item $\forall x<y,\ex z, x<z\land z<y$
    \end{enumerate}
  \end{property}

  1-6 make $\RR$ a field. 1-6 + 7 make $\RR$ an ordered field. 7-8
  makes $\RR$, called by topologists, a \defn{Linear continuum}.

  \begin{theorem}{Well ordering property}
    $\ZZ^+$ has \emph{Well-ordering property}. \ie Every nonempty
    subset of $\ZZ^+$ has a smallest element.
  \end{theorem}
  \begin{proof}
    We first prove that for each $n\in\ZZ^+$, the following statement
    holds: Every nonempty subset of $\set{1,\ldots,n}$ has a smallest
    element.

    Let $A$ be the set of all postive integers $n$ for which this
    theorem holds. Then $A$ contains 1, since if $n=1$, the only
    possible subset is $\set{1}$ itself. Then suppose $A$ contains $n$,
    we show that it contains $n+1$. So let $C$ be a nonempty subset of
    the set $\set{1,\ldots,n+1}$. If $C$ consists of the single
    element $n+1$, then that element is the smallest element of
    $C$. Otherwise, consider the set $C\cap\set{1,\ldots,n}$, which is
    nonempty. Because $n\in A$, this set has a smallest element, which
    will automaticallly be the smallest element of $C$ also. Thus $A$
    is inductive, so we conclude that $A=\ZZ^+$; hence the statement
    is true for all $n\in\ZZ^+$.

    Now we prove the theorem. Suppose that $D$ is a nonempty subset of
    $\ZZ^+$. Choose an element $n$ of $D$. Then the set $A=D\cap[n]$
    is nonempty, so that $A$ has a smallest element $k$. The element
    $k$ is automatically the smallest element of $D$ as well.
  \end{proof}
  \begin{remark}
    I don't really understand the second part of this proof. By
    {\textup{https://proofwiki.org}}, Principle of Mathematical
    Induction, Well-Ordering Principle, and Principle of Complete
    Induction are logically equivalent.
    \link{https://proofwiki.org/wiki/Equivalence_of_Well-Ordering_Principle_and_Induction\#Final_assembly}
  \end{remark}

  \begin{definition}{\defn{Cartesian product}}
    Let $\set{A_1,\ldots,A_m}$ be a faimily of sets indexed with the
    set $\set{1.\ldots,m}$. Let $X=A_1\cup\dots\cup A_m$. We define
    \emph{cartesian product} of this indexed family, denoted by
    $$\prod_{i=1}^mA_i\;\;\text{or}\;\;A_1\times \dots \times A_m,$$
    to be the set of all $m$-tuples $(x_1,\ldots,x_m)$ of elements of
    $X$ such that $x_i\in A_i$ for each $i$.
  \end{definition}
  \begin{remark}
    Indexing function $f:J\to\mathcal{A}$ is surjective but not
    necessarily injective.
  \end{remark}

  \begin{definition}{\textbf{$\omega$-tuple}}
    An \emph{$\omega$-tuple} of elements of set $X$ to be a function
    $$x:\ZZ^+\to X,$$
    \aka \emph{sequence}, or a \emph{infinite sequence}.
  \end{definition}

  \begin{theorem}
    $\set{0,1}^\omega$ is uncountable. (let $X=\set{0,1}$ in the proof.)
  \end{theorem}
  \begin{proof}
    We show that given any function $g:\ZZ^+\to X^\omega$, $g$ is not
    surjective. Four this purpose, let us denote $g(n)$ as
    $(x_{n1},x_{n2},\ldots,x_{nn},\ldots,x_{n\omega})$, where each $x_{ij}$ is
    eather 0 or 1. Then we define any element $y=(y1,\ldots,y_\omega)$
    of $X^\omega$ by letting
    \begin{equation*}
      y_n=
      \begin{cases}
        0 & \text{if $x_{nn}$ = 1}, \\
        1 & \text{if $x_{nn}$ = 0}.
      \end{cases}
    \end{equation*}
    $y$ will differ $g(n)$ for all $n$ by a digit. Therefore
    $y\not\in \operatorname{Im}(g)$.
  \end{proof}
  \begin{remark}
    Note this proof is similar to the proof of uncountableness of
    $[0,1)$ using the vast digit array.
  \end{remark}
  \begin{remark}
    $\set{0,1}^\omega
    \simeq [0,1)$ by $f(a_1,a_2,\ldots)=\sum_{i=1}^\infty{a_i
      2^{-i}}$. (\ie binary decimals). Then we can use the conclusion
    of the uncountableness of $[0,1)$ to prove this directly.
  \end{remark}
  \begin{remark}
    Think of picking a subset of $\ZZ^+$, for each $i\in\ZZ^+$ present in
    the subset, set $a_i=1$, otherwise $a_i=0$. Thus
    $\set{0,1}^\omega$ is just isomorphic to the power set
    $2^{\ZZ^+}$. By cantor's theorem, there is not surjection
    $f : \ZZ^+\to 2^{\ZZ^+}$.
  \end{remark}

  \begin{theorem}
    There is not surjective map $g:A\to 2^A$ for all set $A$. Proof:~\link{https://proofwiki.org/wiki/Cantor's_Theorem}
  \end{theorem}


  \begin{theorem}{\defn{Principle of recursive definition}}
    Let $A$ be a set; let $a_0$ be an element of $A$. Suppose $\rho$
    is a function that assigns, to each function $f$ mapping a
    nonempty section of the positive integers into $A$, an element of
    $A$. Then there exists a unique function
    $$h : \ZZ^+ \to A$$
    such that
    \begin{align*}
      h(1) &= a_0, \\
      h(i) &= \rho(h|\set{1,\ldots,i-1})\;\;\text{for $i>1$}
    \end{align*}
    The formula is called a \emph{recursion formula} for $h$.
  \end{theorem}
  \begin{remark}
    I'm not very clear about this definition. I think the point of
    this definition is to indicate that there is a UNIQUE function
    satisified a recursive definition.
  \end{remark}

  \begin{theorem}
    The following statements about set $A$ are equivalent:
    \begin{enumerate}
    \item There exists an \emph{injective}, not necessarily surjective
      (of course), function $f : \ZZ^+ \to A$.
    \item There exists a bijection of $A$ to a propert subset of $A$.
    \item $A$ is infinite.
    \end{enumerate}
  \end{theorem}
  \begin{example}\mbox{}
    \begin{enumerate}
    \item $f : \ZZ^+\hookrightarrow\RR$.
    \item $f : \RR \to (-1,1)$ by $f(0)=0, f(a)=\frac{1}{a}$.
    \item $\RR$ is infinite.
    \end{enumerate}
  \end{example}

  \begin{definition}{\defn{Axiom of choice}}
    Given a collection $\mathcal{A}$ of disjoint nonempty sets (\ie
    $\forall x,y\in\mathcal{A}, x\cap y=\emptyset$), there
    exists a set $C$ consisting of exactly one element from each
    element of $\mathcal{A}$; that is, $C\subset\bigcup\mathcal{A}$ and for
    each $A\in\mathcal{A}$, the set $C\cap A$ contains a single
    element.

    The set $C$ can be thought of as having been obtained by choosing
    one element from each of the sets in $\mathcal{A}$.
  \end{definition}

  \begin{theorem}\defn{Well-ordering theorem}
    Every set is well-orderable. (assuming axiom of choice)
  \end{theorem}
  \begin{remark}
    This theorem is rather against intuition. As I checked briefly on
    wikipedia and proofwiki, the proof
    \link{https://proofwiki.org/wiki/Well-Ordering_Theorem} is done
    using the axiom of choice and the Princple of Transfinite
    Induction. I can't comprehend the proof since I have no related
    knowledge about transfinite induction schema. The result is also
    known as \defn{Zermelo's
      Theorem}\link{https://proofwiki.org/wiki/Zermelo\%27s_Theorem_(Set_Theory)},
    which is usually worded: Every set of cardinals is well-ordered
    with respect to $\leq$. This is weird to me since doesn't it imply
    that cardinals are equivalent to ordinals?
  \end{remark}

  \begin{definition}\defn{Strict partial order}
    A set $A$, a relation $\prec\;\subset A \times A$ is a \emph{strict
      partial order} if for all elements $a,b,c\in A$,
    \begin{enumerate}
    \item Non-refl: $\neg(x \prec x)$
    \item Trans: $x\prec y\land y\prec z \RA x\prec z$
    \end{enumerate}
  \end{definition}
  \begin{remark}
    Compare to (non-strict) partial order, strict partial order does
    not have reflexivity property ($x\leq x$).
  \end{remark}
  \begin{remark}
    Compare to a simply order relation, this order relation does not
    require totality (or comparability).
  \end{remark}

  \begin{theorem}\defn{The maxium principle}
    A set $A$, a strict partial order $\prec$ on $A$. There exists a
    maxiaml simply ordered subset B.

    Said differently, there exists a subset $B$ of $A$ such that $B$
    is simply ordered by $\prec$ and such that no subset of $A$ that
    properly contains $B$ is simply ordered by $\prec$.
  \end{theorem}
  \begin{remark}
    Think in this way. Draw a acyclic directed graph for $A$: for
    for an element in $A$ draw a vertex, for each pair of vertices,
    draw an edge from $a$ to $b$ iff $a\prec b$ is minimal, \ie
    $\neg\ex c: a\prec c\prec b$. Then we can say
    $a\prec b$ if $a$ is connected to $b$ by the transitivity
    property. This thereom actually says that there exists a maximal
    path in the graph.
  \end{remark}

  \begin{definition}\defn{Upper bound} and \defn{Maximal element}
    Let $A$ be set and $\prec$ be strict partial order on
    $A$. $B\subset A$. An \emph{upper bound} on $B$ is $c\in A$ such
    that $\forall b\in B:b=c\lor b\prec c$. A \emph{maximal element}
    of $A$ is an element $m\in A$ such that $\neg\ex a,m\prec a$.
  \end{definition}
  \begin{remark}
    When we talk about upper bounds of $A$, we are under implication
    of a subset $A$ of a ordered set.
  \end{remark}

  \begin{theorem}\defn{Zorn's Lemma}
    Let $A$ be strictly partially ordered set. If every simply ordered
    subset of $A$ has an upper bound in $A$, then $A$ has a maximal
    element.
  \end{theorem}
  \begin{remark}
    This is a consequence of the maximum principle. It's easy to
    verify the maximum element is the upper bound of the maximal
    simply ordered subset.
  \end{remark}
\end{chapter}

\begin{chapter}{Topological spaces and continuous functions}
  \begin{definition}\defn{Topology}
    A \emph{topology} on a set $X$ is a collection $\mathcal{T}$ of
    subsets of $X$ (\ie $\mathcal{T}\subset 2^X$) having the following properties:
    \begin{enumerate}
    \item $\emptyset \in \mathcal{T}$ and $X\in\mathcal{T}$,
    \item The union of the elements of any subcollection of $\mathcal{T}$ is in
      $\mathcal{T}$,
    \item The intersection of elements of any \underline{finite}
      subcollection of $\mathcal{T}$ is in $\mathcal{T}$.
    \end{enumerate}
  \end{definition}
  \begin{remark}
    After investigation, I consider there is no special meaning for a
    collection differing from a
    set. (ref. \link{http://math.stackexchange.com/a/173002},
    \link{https://proofwiki.org/wiki/Definition:Topology/Definition_1})
  \end{remark}

  \begin{definition}\defn{Topological space}
    A \emph{topological space} is an ordered pair $(X,\mathcal{T})$
    consisting of a set $X$ and a topology $\mathcal{T}$ of
    $X$. Often we omit specfic mention of $\mathcal{T}$ if no
    confusion will arise.
  \end{definition}

  \begin{definition}\defn{Open set}
    If $X$ is a topological space, we say that $U\subset X$ is
    an \emph{open set} of $X$ if $U \in \mathcal{T}$. Using this
    terminology, one can say a topological space is a set $X$ together
    with a set of open sets. Thus the defintion of an open set is the
    same as topology: $\emptyset$ and $X$ both open, arbitrary
    unions and finite intersections of open sets are open.
  \end{definition}

  \begin{definition}
    If $\mathcal{T}=2^A$, then $X$ is called a \defn{Discrete
      topology}; \\
    If $\mathcal{T}=\set{\emptyset, X}$, then $X$ is called a
    \defn{Trivial topology}, or \defn{Indiscrete topology}.
  \end{definition}
  \def\T{\mathcal{T}}
  \begin{definition}\defn{Finite complement topology}
    Let $X$ be a set, $\mathcal{T}_f$ be the collection of all subsets
    $U$ of $X$ such that $X-U$ either is finite or is all of $X$. Then
    $\mathcal{T}_f$ is a topology on $X$, called the \emph{finite
      complement topology}. Both $X$ and $\emptyset$ are in
    $\mathcal{T}_f$ since $X-X$ is finite and $X-\emptyset$ is all of
    $X$.

    Now we show $\mathcal{T}_f$ is a topology. If $\set{U_a}$ is an
    indexed family of nonempty elements of $\T_f$, to show $\bigcup U_a$
    is in $\T_f$, we compute
    $$X-\bigcup U_a = \bigcap(X-U_a)$$
    Which is finite since $X-U_a$ is finite for all $U_i\in\T_f$. To show
    $\bigcap U_i\in\T_f$, we compute
    $$x-\bigcap_{i=1}^n U_i=\bigcup_{i=1}^n(X-U_i)$$
    Note by definition of topology we only need to check finite
    intersection. Then the finite number of unions of finite sets is
    finite thus finite $\bigcup U_i\in\T_f$.
  \end{definition}

  \begin{definition}\defn{Finer}, \defn{Strictly
      finer}, \defn{Coarser}, \defn{Strictly coarser}, and
    \defn{Comparable}
    \begin{table}[H]
      \centering
      \begin{tabular}{l|l|l}
        & set term & topology term \\\hline
        $\T'\supset \T$ & superset & finer, larger, stronger \\
        $\T'\supsetneq \T$ & proper superset(?) & strictly finer  \\
        $\T'\subset \T$ & subset & coarser, larger, weaker   \\
        $\T'\subsetneq \T$ & proper subset & strictly coarser \\
        $\T'\subset \T$ or $\T\subset T'$ & - & comparable
      \end{tabular}
    \end{table}
  \end{definition}
  \begin{remark}
    This is a bit not straightforward at the first sight. Remember
    that the essential of topology is its structure. The more open sets
    we have in a topology, the more fine it is.
  \end{remark}

  \def\B{\mathcal{B}}
  \begin{definition}\defn{Basis}
    A \emph{basis} for a topology $\B\subset 2^X$ (called basis
    elements) such that
    \begin{enumerate}
    \item $\forall x\in X : \exists B\in\B : x\in B$
    \item $\forall x\in B_1 \cap B_2 \RA \ex B_3 \in \B: x\in B_3 \land
      B_3\subset B_1\cap B_2$
    \end{enumerate}

    If $\B$ is a basis, we define \emph{topology $\T$ generated by
      $\B$} to be: we say $U\subset X$ is open set (\ie $U\in\T$), if
    $\forall x\in U: \exists B\in\B: x\in B \land B\subset U$.
  \end{definition}
  \begin{remark}
    The concept of a basis of a topology is rather abstract. I think
    of it in this way. Since we are dealing with intersection in the
    definition of basis, we want to think from top to bottom, in other
    words, from larger sets, to their intersections.
  \end{remark}
  \begin{example}
    Think of this example. In the beginning we have $X=\set{a,b,c}$
    and ${\B=\set{B_1,B_2}}$ where $B_1=\set{a,b}$ and
    $B_2=\set{b,c}$.

    \begin{figure}[H]
      \centering
      \tikz{
        \node (a) [point,label={[name=la] a}] {};
        \node (b) [point,right of=a,label={[name=lb] b}] {};
        \node (c) [point,right of=b,label={[name=lc] c}] {};
        \node (B1) [ellipse,fit=(a)(la)(b),draw,label=below:$B_1$] {};
        \node (B2) [ellipse,fit=(b)(lc)(c),draw,label=below:$B_2$] {};
      }
    \end{figure}

    In this way $\forall x\in X :\ex B\in \B: x\in B$. Now we try to
    satisfy the second condition and we will find that
    $b\in B_1\cap B_2$ is not in any basis element who is a subset of
    $B_1\cap B_2$. Now we add it as follows:

    \begin{figure}[H]
      \centering
      \tikz{
        \node (a) [point,label={[name=la] a}] {};
        \node (b) [point,right of=a,label={[name=lb] b}] {};
        \node (c) [point,right of=b,label={[name=lc] c}] {};
        \node (B1) [ellipse,fit=(a)(la)(b),draw,label=below:$B_1$] {};
        \node (B2) [ellipse,fit=(b)(lc)(c),draw,label=below:$B_2$] {};
        \node (B3) [ellipse,fit=(b)(lb),draw=blue,label={[yshift=-7pt,blue] below:$B_3$}] {};
      }
    \end{figure}

    Now with $\B=\set{B_1,B_2,B_3}$ all above two criterion are
    satisified. Therefore $\B$ is a basis for the topology $\T$:

    \begin{figure}[H]
      \centering
      \tikz{
        \node (a) [point] {};
        \node (b) [point,right of=a] {};
        \node (c) [point,right of=b] {};
        \node (b0) [fit=(b),circle,draw] {};
        \node (ab) [fit=(a)(b0),ellipse,draw,inner sep=0] {};
        \node (bc) [fit=(c)(b0),ellipse,draw,inner sep=0] {};
        \node (abc) [fit=(ab)(bc),ellipse,draw,inner sep=0] {};
      }
    \end{figure}
  \end{example}

  \begin{example}
    Continuing above example. We now verify that $\T$ is generated by
    $\B$ by checking all open sets $U\in\T$ with the rule specified.
    Actually \\
    ${\T=\set{\emptyset, \set{a,b}, \set{b,c}, \set{b},
        \set{a,b,c}}}$. I will only show how $\set{a,b}\in\T$ and how
    $\set{b}\in\T$ and how $\set{a,c}\not\in\T$.

    \begin{enumerate}
    \item $U=\set{a,b}\in\T$: For $a\in U$, take $B=\set{a,b}$, then
      $B\subset U$; the same works for $b$. Therefore $U$ is an open
      set.
    \item $U=\set{b}\in\T$: For $b\in U$, take $B=\set{b}$, then
      $B\subset U$. Therefore $\set{b}$ is an open set. Note that we
      cannot take $B=\set{a,b}$, since $\set{a,b}\not\subset \set{b}$.
    \item $U=\set{a,c}\not\in\T$: For $a\in U$, we cannot find a
      $B\in\B$ that contains $a$ and is a subset of $\set{a,c}$. Since
      the only possible subsets of $\set{a,c}$ are $\set{\emptyset,
        \set{a}, \set{c}, \set{a,c}}$ and neither is a basis
      element. Therefore $\set{a,c}$ is not an open set.
    \end{enumerate}
  \end{example}
  \begin{remark}
    Let $\B$ be set of all one-point subsets of $X$, then it is a
    basis for the discrete topology on $X$.
  \end{remark}
  \begin{remark}
    So my understanding of a basis of a topology is like the generator
    of the topology that satisfy the existence of intersection. So
    first of all we have to note that $\B\subset\T$.

    Using the process from the book, we can verify it. Take $J$ to be
    an indexed family of $\B$.

    It's easy to show that $\bigcup_{\alpha\in J} B_\alpha$ is in $\T$
    if $\forall \alpha\in J : B_\alpha\in\T$.

    It's easier to to show that $\bigcap_{\alpha\in J} B_\alpha$ is in
    $\T$ if $\forall \alpha\in J: B_\alpha\in\T$ since it's specified
    in the criteria for $\B$ to be a basis. (Actually basis does more
    than finite intersection. We can prove using induction that the
    intersection of any countable number of basis is in $\T$.

    Also, $\emptyset\in\T$ will be vacuously true no matter what $\B$
    we pick.
  \end{remark}
  \begin{theorem}
    A more visual-able theorem. $\T$ is the all unions of elements in
    $\bigcup_{B\in\B} B$.
  \end{theorem}
  \begin{proof}
    Given $B=\bigcup_{\alpha\in K} B_\alpha$, since $B_\alpha\in\T$
    and for any $x,y\in\T$, $x\cup y\in\T$. Thus $B\in\T$. Conversely,
    given $U\in\T$, choose for each $x\in U$, $\ex B_x\in\B$ and
    $U=\bigcup_{x\in U} B_x$. Therefore $U$ is some union of elements
    in $\B$.
  \end{proof}
  \begin{theorem}
    Let $X$ be a topological space. $\mathcal{C}\subset \T$ such that
    for each open set $U\in\T$ and each $x\in U$, there is an element
    $C\in\mathcal{C}$ such that $x\in C\subset U$. Then $\mathcal{C}$
    is a basis for $X$.
  \end{theorem}
  \begin{theorem}
    Let $\B$ and $\B'$ be bases for $\T$ and $\T'$, respectively. Then
    the following are equivalent:
    \begin{enumerate}
    \item $\T'$ is finer than $\T$
    \item
      $\forall x\in X, B\in\B : x\in B \RA \exists B'\in\B' : x\in B'
      \subset B$.
    \end{enumerate}
  \end{theorem}
  \begin{remark}
    Think of a finer topology to be a set with more elements, while
    inclusively. Then it works just like the defintion of
    ``subsets''. While we are now not talking about the topologies but
    the bases, so we only care about the elements in the bases.

    The book uses the concept of a gravel. So the pebbles forms a
    basis of a topology. When they get smashed into dust, they form
    the basis of a new topology, while finer. And the dust particles
    was contained inside a pebble, as says the criterion.
  \end{remark}
  \begin{example}
    One can be demonstrated that topology on $\RR$ generated by open
    circles is the same topology generated by rectangles. Below
    diagram shows this:

    \begin{figure}[H]
      \centering
      \begin{tikz}
        \node (x) [point,label={[name=xl] below:x}] {};
        \node [right of=x,fit=(x)(xl),circle,draw] {a};
      \end{tikz}
    \end{figure}
  \end{example}

  \begin{definition}\defn{Order topology}
    Assuming we knowing about all open/closed/half-open interval
    concepts, \emph{over a, possibly not continuous, set}.

    Let $X$ be a set with linear order relation with $<$. Assume
    $|X|>1$. Let $\B$ be defined as a set of all subsets of $X$ of the
    following types:
    \begin{enumerate}
    \item All valid $(a,b)$
    \item For $a_0$ to be the minimal element, $[a_0,b)$ (if any)
    \item For $b_0$ to be the maximal element, $(a,b_0]$ (if any)
    \end{enumerate}

    Then $B$ is a basis for a topology on $X$, called \emph{order
      topology}.
  \end{definition}
  \begin{example}
    $\RR$, $\RR\times\RR$ in dict order, $\ZZ^+$
  \end{example}
  \begin{definition}\defn{Ray}, \defn{Open ray}, \defn{Closed ray}
    \begin{align*}
      (a,+\infty)&=\set{x\mid x>a} \\
      (-\infty,a)&=\set{x\mid x<a} \\
      [a,+\infty)&=\set{x\mid x\geq a} \\
      (-\infty,a]&=\set{x\mid x\leq a}
    \end{align*}
  \end{definition}

  \begin{definition}\defn{Product topology}
    Product topology on topology spaces $X$ and $Y$, denoted as
    $X\times Y$ is the topology with the basis $\B$ who elements are
    in form of $U\times V$ where $U\in\T_X, V\in\T_Y$. (Obviously and
    omitted by book the underlying set is just $X\times Y$)
  \end{definition}
  \begin{example}
    open sets on $\RR\times \RR=\RR^2$.
  \end{example}
  \begin{theorem}
    If $\B$ is a basis for topology  of $X$ and $\mathcal{C}$ is a
    basis for the topology of $Y$, then the collection
    $\mathcal{D}=\set{B\times C\mid B\in \mathcal{B}\land
      C\in\mathcal{C}}$
    is a basis for $X\times Y$.
  \end{theorem}
  \begin{definition}\defn{Projection}
    $$\pi_1:X\times Y\to X,\quad\pi_2:X\times Y\to Y$$
    Projection functions are onto.
  \end{definition}

\end{chapter}

\printindex

\end{document}

%%% Local Variables:
%%% mode: latex
%%% TeX-master: t
%%% End:
