\documentclass{report}

\usepackage[utf8]{inputenc}
\usepackage{imakeidx}
\makeindex[columns=2,title=Alphabetical Index]

\usepackage{amsmath}
\usepackage{amsthm}
\usepackage{amssymb}

\usepackage{hyperref}

\usepackage{xspace}
\usepackage{enumitem}

\newtheorem*{remark}{Remark}

\theoremstyle{definition}
\newtheorem{theorem}{Theorem}[chapter]
\newtheorem{definition}[theorem]{Definition}
\newtheorem{example}[theorem]{Example}
\newtheorem{property}[theorem]{Property}

\newcommand{\defn}[1]{\textbf{#1}\label{#1}\index{#1}}
\newcommand{\set}[1]{\ensuremath\{#1\}}

\newcommand{\ie}{\textit{i.e.}\xspace}
\newcommand{\aka}{\textit{a.k.a.}\xspace}
\newcommand{\eg}{\textit{e.g.}\xspace}

\newcommand{\RA}{\ensuremath\Rightarrow}

\newcommand{\ZZ}{\mathbb{Z}}
\newcommand{\RR}{\mathbb{R}}

\newcommand{\A}{$A$\xspace}
\newcommand{\B}{$B$\xspace}
\newcommand{\X}{$X$\xspace}
\renewcommand{\a}{$a$\xspace}
\renewcommand{\b}{$b$\xspace}
\newcommand{\x}{$x$\xspace}

\newcommand{\ex}{\ensuremath\exists}



\begin{document}
\newcommand*{\link}[1]{\href{#1}{(\underline{link})}%
  \footnote{\url{#1}}}


\title{Notes for James R. Munkres' Topology (2E)}
\author{Shou}
\date{\today}

\maketitle
\tableofcontents

\setcounter{chapter}{-1}
\begin{chapter}{Structure and reading plans}
Ch 1-8 is the part I, mainly for common topology. The part II includes
ch 9-14, that depends on ch 1-4, is about algebraic topology.

My plan is to read through ch 1-4 very quickly, within a weekend, and
then I will start reading ch 9+ simultaneously with W.S.Massey's
Agelbraic topology: An induction.

Finally I wish I could finish all ch 1-8 and also some parts after ch 9.
\end{chapter}

\begin{chapter}{Set theory and logic}
  \begin{definition}{\defn{Order relation}}
    rel $C$ on set $A$ is called \emph{order relation} if
    \begin{enumerate}
    \item comparability, \ie
      $\forall x,y\in A, x\not=y\RA x C y\lor y C x$
    \item non-refl, \ie $\forall x, \neg(x C x)$
    \item trans, \ie $\forall x C y\land y C z, x C z$
    \end{enumerate}
    (\aka linear order)
  \end{definition}

  \begin{definition}{\defn{Open interval}}
    if $X$ is a set and $<$ is an order rel, and if $a<b$ we use
    notation $(a,b)$ to denote $\set{x\in X\mid a<x<b}$, called
    \emph{open interval}.

    If $(a,b) = \emptyset$, then $a$ is called \emph{immediate
      precessor} of $b$ and $b$ called \emph{immediate successor} of
    $a$.
  \end{definition}
  \begin{remark}
    It makes more sense on $X$ is a discrete set. Since if
    $(a,b)$ is an open interval in $\RR$, $(a,b) = \emptyset
    \RA a = b$ which makes no sense on $a$ as an immediate precessor
    of $b$.
  \end{remark}

  \begin{definition}{\defn{Order Type}}
    if $A$ and $B$ are two sets with $<_A$ and $<_B$. We say that \A
    and \B have same \emph{order type} if $\ex f : A \to B$ that
    preserves order, \ie $$a_1 <_A b_1 \RA f(a_1) <_B f(b_1)$$
  \end{definition}
  \begin{remark}
    It's just a generalization of monotone function.
  \end{remark}

  \begin{definition}{\defn{Dictionary order relation}}
    if \A,\B are two sets with $(<_A,<_B)$, defn an order for
    $A\times B$ by defining $$a_1\times b_1 < a_2\times b_2$$
    if $a_1<_A a_2$, or if $a_1=a_2\;\land\;b_1<_B b_2$.
  \end{definition}

  \begin{definition}{\defn{LUB property}/\defn{GLB property}}
    For $A$ and $<_A$, we say \A has \emph{LUB property} if
    $$\forall A_0\subset A,A\not=\emptyset\land\ex \text{upper bound for }
    A_0 \RA \ex \text{lub$\{A_0\}$}\in A$$
  \end{definition}
  \begin{example}
    $A=(-1,1)$. \eg $X=\set{1-\frac{1}{n}\mid n\in\ZZ^+}$ does not have an
    upper bound, thus vacuously true. $\set{-\frac{1}{n}\mid
      n\in\ZZ^+}$ has upper bound of any number in $[0,1)\subset A$, and
    $\operatorname{lub}(X)=0\in (-1,1)$.
  \end{example}
  \begin{example}
    Counterexample. $A=(-1,0)\cup(0,1)$. $\set{-\frac{1}{n}\mid n\in\ZZ^+}$ has
    upper bound of any $(0,1)\subset A$, while
    $\operatorname{lub}(X)=0\not\in A$.
  \end{example}
  \begin{remark}
    The completeness property of $\RR$ as an axiom derives this property.
  \end{remark}

  \begin{property}{$\RR$ field}

    \textbf{Algebraic properties}
    \begin{enumerate}
    \item assoc: $(x+y)+z=x+(y+z); (xy)z=x(yz)$
    \item comm: $x+y=y+x; xy=yx$
    \item id: $\ex!0, x+0=x$; $\ex!1,x\not=0\RA x1=x$
    \item inv: $\forall x,\ex!y,x+y=0$; $\forall x\not=0,\ex!y,xy=1$
    \item distr: $x(y+z)=xy+xz$
    \end{enumerate}

    \textbf{Mixed algebraic and order property}
    \begin{enumerate}[resume]
    \item $x>y\RA x+z>y+z$; $x>y\land z>0\RA xz>yz$
    \end{enumerate}

    \textbf{Order properties}
    \begin{enumerate}[resume]
    \item $<$ has LUB property
    \item $\forall x<y,\ex z, x<z\land z<y$
    \end{enumerate}
  \end{property}

  1-6 make $\RR$ a field. 1-6 + 7 make $\RR$ an ordered field. 7-8
  makes $\RR$, called by topologists, a \defn{linear continuum}.

  \begin{theorem}{Well ordering property}
    $\ZZ^+$ has \emph{Well-ordering property}. \ie Every nonempty
    subset of $\ZZ^+$ has a smallest element.
  \end{theorem}
  \begin{proof}
    We first prove that for each $n\in\ZZ^+$, the following statement
    holds: Every nonempty subset of $\set{1,\ldots,n}$ has a smallest
    element.

    Let $A$ be the set of all postive integers $n$ for which this
    theorem holds. Then \A contains 1, since if $n=1$, the only
    possible subset is $\set{1}$ itself. Then suppose \A contains $n$,
    we show that it contains $n+1$. So let $C$ be a nonempty subset of
    the set $\set{1,\ldots,n+1}$. If $C$ consists of the single
    element $n+1$, then that element is the smallest element of
    $C$. Otherwise, consider the set $C\cap\set{1,\ldots,n}$, which is
    nonempty. Because $n\in A$, this set has a smallest element, which
    will automaticallly be the smallest element of $C$ also. Thus $A$
    is inductive, so we conclude that $A=\ZZ^+$; hence the statement
    is true for all $n\in\ZZ^+$.

    Now we prove the theorem. Suppose that $D$ is a nonempty subset of
    $\ZZ^+$. Choose an element $n$ of $D$. Then the set $A=D\cap[n]$
    is nonempty, so that $A$ has a smallest element $k$. The element
    $k$ is automatically the smallest element of $D$ as well.
  \end{proof}
  \begin{remark}
    I don't really understand the second part of this proof. By
    {\textup{https://proofwiki.org}}, Principle of Mathematical
    Induction, Well-Ordering Principle, and Principle of Complete
    Induction are logically equivalent.
    \link{https://proofwiki.org/wiki/Equivalence_of_Well-Ordering_Principle_and_Induction\#Final_assembly}
  \end{remark}

  \begin{definition}{\defn{Cartesian product}}
    Let $\set{A_1,\ldots,A_m}$ be a faimily of sets indexed with the
    set $\set{1.\ldots,m}$. Let $X=A_1\cup\dots\cup A_m$. We define
    \emph{cartesian product} of this indexed family, denoted by
    $$\prod_{i=1}^mA_i\;\;\text{or}\;\;A_1\times \dots \times A_m,$$
    to be the set of all $m$-tuples $(x_1,\ldots,x_m)$ of elements of
    $X$ such that $x_i\in A_i$ for each $i$.
  \end{definition}
  \begin{remark}
    Indexing function $f:J\to\mathcal{A}$ is surjective but not
    necessarily injective.
  \end{remark}

  \begin{definition}{\textbf{$\omega$-tuple}}
    An \emph{$\omega$-tuple} of elements of set $X$ to be a function
    $$x:\ZZ^+\to X,$$
    \aka \emph{sequence}, or a \emph{infinite sequence}.
  \end{definition}

  \begin{theorem}
    $\set{0,1}^\omega$ is uncountable. (let $X=\set{0,1}$ in the proof.)
  \end{theorem}
  \begin{proof}
    We show that given any function $g:\ZZ^+\to X^\omega$, $g$ is not
    surjective. Four this purpose, let us denote $g(n)$ as
    $(x_{n1},x_{n2},\ldots,x_{nn},\ldots,x_{n\omega})$, where each $x_{ij}$ is
    eather 0 or 1. Then we define any element $y=(y1,\ldots,y_\omega)$
    of $X^\omega$ by letting
    \begin{equation*}
      y_n=
      \begin{cases}
        0 & \text{if $x_{nn}$ = 1}, \\
        1 & \text{if $x_{nn}$ = 0}.
      \end{cases}
    \end{equation*}
    $y$ will differ $g(n)$ for all $n$ by a digit. Therefore
    $y\not\in \operatorname{Im}(g)$.
  \end{proof}
  \begin{remark}
    Note this proof is similar to the proof of uncountableness of
    $[0,1)$ using the vast digit array.
  \end{remark}
  \begin{remark}
    $\set{0,1}^\omega
    \simeq [0,1)$ by $f(a_1,a_2,\ldots)=\sum_{i=1}^\infty{a_i
      2^{-i}}$. (\ie binary decimals). Then we can use the conclusion
    of the uncountableness of $[0,1)$ to prove this directly.
  \end{remark}
  \begin{remark}
    Think of picking a subset of $\ZZ^+$, for each $i\in\ZZ^+$ present in
    the subset, set $a_i=1$, otherwise $a_i=0$. Thus
    $\set{0,1}^\omega$ is just isomorphic to the power set
    $2^{\ZZ^+}$. By cantor's theorem, there is not surjection
    $f : \ZZ^+\to 2^{\ZZ^+}$.
  \end{remark}

  \begin{theorem}
    There is not surjective map $g:A\to 2^A$ for all set $A$. Proof:~\link{https://proofwiki.org/wiki/Cantor's_Theorem}
  \end{theorem}


  \begin{theorem}{\defn{Principle of recursive definition}}
    Let $A$ be a set; let $a_0$ be an element of $A$. Suppose $\rho$
    is a function that assigns, to each function $f$ mapping a
    nonempty section of the positive integers into $A$, an element of
    $A$. Then there exists a unique function
    $$h : \ZZ^+ \to A$$
    such that
    \begin{align*}
      h(1) &= a_0, \\
      h(i) &= \rho(h|\set{1,\ldots,i-1})\;\;\text{for $i>1$}
    \end{align*}
    The formula is called a \emph{recursion formula} for $h$.
  \end{theorem}
  \begin{remark}
    I'm not very clear about this definition. I think the point of
    this definition is to indicate that there is a UNIQUE function
    satisified a recursive definition.
  \end{remark}

  \begin{theorem}
    The following statements about set $A$ are equivalent:
    \begin{enumerate}
    \item There exists an \emph{injective}, not necessarily surjective
      (of course), function $f : \ZZ^+ \to A$.
    \item There exists a bijection of $A$ to a propert subset of $A$.
    \item $A$ is infinite.
    \end{enumerate}
  \end{theorem}
  \begin{example}\mbox{}
    \begin{enumerate}
    \item $f : \ZZ^+\hookrightarrow\RR$.
    \item $f : \RR \to (-1,1)$ by $f(0)=0, f(a)=\frac{1}{a}$.
    \item $\RR$ is infinite.
    \end{enumerate}
  \end{example}

  \begin{definition}{\defn{Axiom of choice}}
    Given a collection $\mathcal{A}$ of disjoint nonempty sets (\ie
    $\forall x,y\in\mathcal{A}, x\cap y=\emptyset$), there
    exists a set $C$ consisting of exactly one element from each
    element of $\mathcal{A}$; that is, $C\subset\bigcup\mathcal{A}$ and for
    each $A\in\mathcal{A}$, the set $C\cap A$ contains a single
    element.

    The set $C$ can be thought of as having been obtained by choosing
    one element from each of the sets in $\mathcal{A}$.
  \end{definition}
\end{chapter}

\printindex

\end{document}

%%% Local Variables:
%%% mode: latex
%%% TeX-master: t
%%% End:
